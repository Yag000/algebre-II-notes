\subsection{Théorème de Cayley-Hamilton}


\begin{theorem}
	Soit $u \in \LE$ et $\dim E = n$. Alors le polynôme caractéristique de $u$ est un polynôme annulateur de $u$.
	$$ \chi_u(u) = 0_{\LE}$$
\end{theorem}


\begin{proof}
	Soit $x \in E$ non nul. $E_{u,x} = \vect(u^k(x))$ stable par $u$.

	%TODO
\end{proof}


\subsubsection{Sous espaces caractéristiques}

\begin{coro}
	Soit $u \in \LE$ tel que $\chi_u$ est scindé, i.e. $\chi_u = \prod_{i=1}^r (X - \lambda_i)^{m_i}$.
	Alors $E$ est la somme directe des sous espaces caractéristiques de $u$. En particulier, a dimension du
	sous espace caractéristique est égale a la multiplicité de la valeur propre associée.
\end{coro}

\begin{proof}
	$\chi_u(X) = \prod_{i=1}^r (X - \lambda_i)^{m_i}$ où les $\lambda_i$ sont distincts, avec $P_i(X) = (X - \lambda_i)^{m_i}$.
	Les $P_i$ sont premiers entre eux. D'après le lemme des noyaux, on a que $E = \bigoplus_{i=1}^r \ker P_i(u)$.

	Soit $v_j = \restr u {\ker P_j(u)}$. On a que $\chi_u = \prod_{j=1}^r \chi_{v_j}$. Or la seule racine de $P_i$ est $\lambda_i$,
	$u$ annule $P_i$ et $\deg P_i = m_i$. Donc $m_i$ est bien la dimension du sous espace caractéristique associé à $\lambda_i$.
	$$ \deg \chi_{v_i} = \dim (\ker P_i(u)) = m_i$$
\end{proof}

\subsubsection{Multiplicités}

$u \in \LE$ , $\lambda$ valeur propre, $\dim E = n$.

$m_a(\lambda)$ la multiplicité algébrique de $\lambda$ est sa multiplicité en tant que racine du polynôme caractéristique $\chi_u$.
$$ M = \begin{pmatrix}
		1 & 1 & 0 \\ 0 & 1 & 0 \\ 0 & 0 & 1
	\end{pmatrix},  \ m_a(1) = 3$$

$m_m(\lambda)$ la multiplicité minimale de $\lambda$ est la multiplicité de $\lambda$ en tant que racine du polynôme minimal $\mu_u$.
Dans l'exemple précédent, $m_m(1) = 2$.

On a que $\mu_u \mid \chi_u$ et donc $m_m(\lambda) \leq m_a(\lambda)$.

$\mu_u \in  \set {(X-1), (X-1)^2, (X-1)^3}$

$M - I_3 = \begin{pmatrix}
		0 & 1 & 0 \\ 0 & 0 & 0 \\ 0 & 0 & 0
	\end{pmatrix} \neq 0$ Donc $\mu_u \neq X - 1$.
$(M - I_3)^2 = 0$ donc $\mu_u = (X-1)^2$, d'où $m_m(1) = 2$.


$m_g(\lambda)$ la multiplicité géométrique de $\lambda$ et est la dimension du sous espace caractéristique associé à $\lambda$.

$E_1 (M) = \kre (u-id) = \vect (e_1, e_3)$ donc $m_g(1) = 2$.


\begin{prop}
	Soit $u \in \LE$ et $\lambda$ une valeur propre de $u$. On sait que $ 1 \geq m_g(\lambda) \geq m_a(\lambda)$ et $1 \geq m_m(\lambda) \geq m_a(\lambda)$.

	Ceci est une consequence du lemme suivant.
\end{prop}

\begin{lemma}
	Soit $u \in \LE$. Les polynômes $\mu_u$ et $\chi_u$ ont les mêmes facteurs irréductibles.
\end{lemma}

\begin{proof}
	Le théorème de Cayley-Hamilton nous dit que $\mu_u \mid \chi_u$. Soit $M = \text {mat} u$.
	%TODO
\end{proof}








