\subsection{Matrices compagnons}



\begin{definition}
	La matrice compagnon ou matrice de Frobenius d'un polynôme unitaire $P(X)  = X^n - \sum_{k=0}^{n-1} a_k X^k$ est la matrice
	$$ C_P = \begin{pmatrix}
			0      & 0      & \ldots & 0      & a_0     \\
			1      & 0      & \ldots & 0      & a_1     \\
			0      & 1      & \ldots & 0      & a_2     \\
			\vdots & \vdots & \ddots & \vdots & \vdots  \\
			0      & 0      & \ldots & 1      & a_{n-1}
		\end{pmatrix}$$
\end{definition}

\begin{remarque}
	Soit $u$ l'endomorphisme de $E$ tel que $C_p$ soit la matrice de $u$ dans une base $\mathcal{B} = (e_1, \ldots, e_n)$.
	\begin{itemize}
		\item $u(e_1) = e_2$
		\item $u(e_j)$ = $e_{j+1}$ pour $j \in \{2, \ldots, n-1\}$
		\item $u(e_n) = \sum_{k=0}^{n-1} a_k e_{k+1}$
	\end{itemize}

	%TODO add as a } 

	$u^k(e_i) = e_{k+1}$ pour $i \in \{1, \ldots, n-1\}$
	$$ (u^n - \sum_{k=0}^{n-1} a_k u^k)(e_1) = 0_E$$
\end{remarque}

\begin{remarque}
	Un endomorphisme $u \in \LE$ est cyclique s'il existe une base $\mathcal{B}$ de $E$ telle que la matrice
	de $u$ dans cette base soit une matrice compagnon.
\end{remarque}

\begin{proof}
	%TODO finish proof
	\begin{itemize}
		\item Soit $x \in E$ tel que $E_{u,x} = E$, et on choisit $\mathcal{B} = (x, u(x), \dots, u^{n-1}(x))$.
	\end{itemize}
\end{proof}

\subsubsection{Polynôme caractéristique }

\begin{prop}
	Le polynôme caractéristique de $C_P$, la matrice compagnon associée à $P$, est $P$.
\end{prop}

\begin{proof}
	$\Delta(a_1, \ldots, a_n)(X) = \det (XI_n - C_P) =
		\begin{vmatrix}
			X      & 0      & \ldots & 0      & -a_0      \\
			-1     & X      & \ldots & 0      & -a_1      \\
			0      & -1     & \ldots & 0      & -a_2      \\
			\vdots & \vdots & \ddots & \vdots & \vdots    \\
			0      & 0      & \ldots & -1     & X-a_{n-1}
		\end{vmatrix}$

	On développe par rapport à la premiere ligne.

	%TODO: Explanation ??
	$$ \Delta(a_1, \ldots, a_n)(X) = (-1)^{1+1}X\Delta(a_1, \ldots, a_n-1)(X) + (-1)^{1+n}(-1)^{n-1}a_0$$
	et donc $\Delta(a_1, \ldots, a_n)(X) = X\Delta(a_1, \ldots, a_n-1)(X) - a_0 $
	Et par itération on obtient $\Delta(a_1, \ldots, a_n)(X) = X^n - \sum_{k=0}^{n-1} a_k X^k$
\end{proof}



\begin{prop}
	Soit $u \in \LE$ cyclique et $x \in E$ tel que $E_{u,x} = E$. On choisit les $a_k \in $ tels que
	$$u^n(x) = \sum_{k=0}^{n-1} a_k u^k(x)$$
	Alors $\chi_u (X) = X^n - \sum_{k=0}^{n-1} a_k X^k$
\end{prop}

\begin{remarque}
	Un corollaire direct de cet énoncé dit que tout polynôme unitaire de degré $n$ dans $\K[X]$ est le polynôme caractéristique
	d'une matrice de $\mathcal{M}_n(\K)$.
\end{remarque}

\begin{theorem}[Cas cyclique du théorème de Cayley-Hamilton]
	Soit $u \in \LE$ cyclique, alors $\chi_u $ le polynôme caractéristique de $u$ est un
	polynôme annulateur de $u$.
\end{theorem}

\begin{proof}
	Cela découle du fait que dans $(x, u(x), \ldots, u^{n-1}(x))$ la matrice de $u$ est une matrice compagnon associée à un $P \in \K[X]$.
	$P$ est construit pour que $P(u)(x) = 0_E$.\\
	On a $P(u) = 0_{\LE}$ et $\chi_u = P$.
\end{proof}


\begin{example}
	Une condition nécessaire et suffisante pour qu'un endomorphisme nilpotent soit cyclique est
	que son indice de nilpotence soit égal à $n = \dim E$.
\end{example}
