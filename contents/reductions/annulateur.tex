\subsection{Polynômes annulateurs et polynôme minimal}


\begin{definition}
	Soit $u \in \LE$ et $P \in K[X]$. On dit que $P$ est un polynôme annulateur de $u$ si $P(u) = 0_{\LE}$
\end{definition}


\begin{prop}
	Tout endomorphisme dans un espace vectoriel de dimension finie admet un polynôme annulateur non nul.
\end{prop}


\begin{proof}
	$\set{u^k \mid 0 \leq k \leq n^2}$ est une famille liée car $\dim \LE = n^2$.\\
\end{proof}


\begin{remarque}
	Le théorème de Cayley-Hamilton nous dira que le polynôme caractéristique est un polynôme annulateur.
\end{remarque}

\begin{remarque}
	Si $E$ n'est pas de dimension finie, cela n'est pas vrai en general:

	Soit $D : K[X] \to K[X]$ l'application qui à un polynôme associe sa dérivée. Alors $D$ n'a pas de polynôme annulateur.\\
	Soit $Q = \sum\limits_{i=0}^n a_iX^i \in K[X]$.
	$$Q(D)(P) = \sum\limits_{k=0}^n ia_i P^{(k)}$$
\end{remarque}



\begin{prop}
	L'ensemble $I_u$ des polynômes annulateurs de $u \in \LE$ est un idéal de $K[X]$.
	Si $I_u \neq \set{0}$, alors il admet un générateur unique appelé polynôme minimal de $u$ et noté $\mu_u$.
\end{prop}


\begin{definition}
	Un endomorphisme $u$ est dit nilpotent s'il existe $n \in \N$ tel que $u^n = 0_{\LE}$.\\
	Dans ce cas, le polynôme minimal de $u$ est de la forme $X^n$ et on appelle $n$ l'indice de nilpotence de $u$.
\end{definition}

\begin{remarque}
	Si $u$ admet un polynôme minimal $\mu_u$ de degré $d$, alors $\dim K[u] = d$.
\end{remarque}

\begin{proof}
	$$\phi : \begin{array}{rcl}
			K[X] & \to     & K[u] \\
			P    & \mapsto & P(u)
		\end{array}$$
	$$\ker \phi = <\mu_U(u)>$$
	$$ K[u] \cong K[X] / <\mu_U>$$
	Donc $\dim K[u] = \deg \mu_U = d$
	%TODO
\end{proof}

\begin{exemple}
	Si $u$ est une homothétie de rapport $\lambda$, autrement dit si $u = \lambda id_E$, alors $\mu_u = X - \lambda$.
\end{exemple}

\begin{remarque}
	On dit que $p$ est un projecteur \ssi ${p}^2 = p$.
	$\mu_p = X^2 - X$ sauf si $p = id_E$ ou $p = 0_{\LE}$.
\end{remarque}

\begin{prop}
	Soit $u \in \LE$ admettant un polynôme minimal $\mu_u$, alors $u$ est inversible \ssi $\mu_u(0) \neq 0$
\end{prop}

\begin{proof}
	%TODO
\end{proof}
