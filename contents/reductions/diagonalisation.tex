\subsection{Diagonalisation}

\subsubsection{Critères de diagonalisation}

\begin{remarque}
	Comme les vecteurs colonnes d'une matrice sont les images des vecteurs de
	la base, si la matrice est diagonale, alors la base choisie est une base de
	vecteurs propres de la matrice.
\end{remarque}


\begin{definition}
	Un endomorphisme est diagonalisable s'il existe une base dans laquelle sa matrice est diagonale, autrement dit, il
	existe une base de vecteurs propres pour cet endomorphisme ou encore, l'espace $E$ est la somme directe des sous-espaces propres.
\end{definition}

\begin{rappel}
	Soit $A, B \in \Mn n K$, on dit que $A$ et $B$ sont semblables s'il existe
	$P \in \mathcal{M}_n(K)$ inversible telle que $A = PBP^{-1}$. C'est une relation d'équivalence.
\end{rappel}

\begin{example}
	Si $M$ est diagonalisable et $M$ admet une seule valeur propre, alors il existe $\lambda \in K$ tel que $M = \lambda I_n$.
	En effet, $M$ et $I_n$ sont semblables et donc égales car la classe d'équivalence de $I_n$ est $\{I_n\}$.
\end{example}



\begin{prop}
	Soit $u \in \LE$ et $n = \dim E$. Si $u$ admet $n$ valeurs propres distinctes, alors $u$ est diagonalisable.
	En particulier, une matrice triangulaire supérieure avec des coefficients diagonaux 2 à 2 distincts est diagonalisable.
\end{prop}

\begin{example}
	$ \begin{pmatrix}
			1 & 2 \\
			0 & 3
		\end{pmatrix} $ est diagonalisable car ses valeurs propres sont 1 et 3.\\
	$\begin{pmatrix}
			1 & 2 \\ 0 & 1
		\end{pmatrix} $ n'est pas diagonalisable, sinon elle serait semblable à la matrice identité.
\end{example}

On a montré qu'une famille de $n$ vecteurs propres associés à des valeurs propres distinctes est une famille libre.

\begin{coro}
	Soit $u \in \LE$. Si $\chi_u$, le polynôme caractéristique de $u$, est scindé à racines simples, alors $u$ est diagonalisable.
\end{coro}

\begin{proof}
	Si $\lambda$ est une racine de $\chi_u$, alors $\lambda$ est une valeur propre de $u$. Donc on a $n$ valeurs propres distinctes et donc $u$ est diagonalisable.
\end{proof}

\begin{theorem}
	Soit $u \in \LE$ et $n = \dim E$. Les assertions suivantes sont équivalentes:
	\begin{enumerate}
		\item $u$ est diagonalisable.
		\item $u$ admet un polynôme annulateur scindé à racines simples.
		\item $\mu_u$, le polynôme minimal de $u$ est scindé à racines simples.
	\end{enumerate}
\end{theorem}

\begin{proof}
	\begin{itemize}
		\item $(2) \implies (3)$: Si $P$ est scindé à racines simples annulateur alors $\mu_u \mid P$ et donc on a $(3)$.
		\item $(3) \implies (1)$: On a $\mu_u \prod_{i=1}^n (X - \lambda_i)$ avec les $\lambda_i \in K$. On applique le lemme des noyaux.
		      $$ E = \ker \mu_u(u) \oplus \bigoplus_{i=1}^n \ker(u - \lambda_i \id_E) $$
		      Et donc $u$ est diagonalisable.
		\item $(3) \implies (2)$: $\mu_u$ est annulateur de $u$ et donc on a $(2)$.
		\item $(1) \implies (2)$: Dans une base des vecteurs propres, la matrice de $u$ s'écrit $D = \begin{pmatrix}
				      \lambda_1 & 0         & \cdots & 0         \\
				      0         & \lambda_2 & \cdots & 0         \\
				      \vdots    & \vdots    & \ddots & \vdots    \\
				      0         & 0         & \cdots & \lambda_n
			      \end{pmatrix}$, où les $\lambda_i$ ne sont pas forcément distincts.
		      Soit $P \in \K[X]$, dans cette base $P(u)$ a comme matrice
		      $$P(D) = \begin{pmatrix}
				      P(\lambda_1) & 0            & \cdots & 0            \\
				      0            & P(\lambda_2) & \cdots & 0            \\
				      \vdots       & \vdots       & \ddots & \vdots       \\
				      0            & 0            & \cdots & P(\lambda_n)
			      \end{pmatrix}$$
		      On choisit $P = \prod_{i=1}^n (X - \lambda_i)$, alors $P(D) = 0$ et donc $\chi_u$ est annulateur de $u$.

		      Une autre manière:
		      \begin{eqnarray*}
			      u \text{ diagonalisable } \implies E &=& \bigoplus_{i=1}^n \ker(u - \lambda_i \id_E) \\
			      &=& \ker P(u) \\
			      &\implies& P(u) = 0
		      \end{eqnarray*}
	\end{itemize}
\end{proof}

\begin{example}
	Les projecteurs sont diagonalisables car ils annulent $X(X-1)$.

	Les symétries vectorielles (qui vérifient $s^2 = \id$) sont diagonalisables
	si $\car K \neq 2$, car elles annulent $X^2 - 1 = (X-1)(X+1)$ et $-1 \neq 1$.
\end{example}


\begin{example}
	Soit $a \in \Mn q K$, $C \in \Mn {n-p} K$ et $B \in \mathcal{M}_{q, n-p}(K)$. %TODO Change to capital a 
	On choisit $A = \begin{pmatrix} a & B \\ 0 & C \end{pmatrix}$.\\
	$M$ diagonalisable $\implies$ $a$ et $C$ diagonalisables.\\
	$P(M) = \begin{pmatrix} P(a) & * \\ 0 & P(C) \end{pmatrix}$. Si $P$ scindé à racines simples annule $M$, alors $P$ annule $a$ et $C$.
	D'après le théorème, $a$ et $C$ sont diagonalisables.
	La réciproque est fausse.
\end{example}


\begin{prop}
	Soit $u \in \LE$ diagonalisable et $F$ stable par $U$, alors $\restr u F $ est diagonalisable.
\end{prop}

\begin{proof}
	$\exists P$ scindé à racines simples tel que $P(u) = 0$, donc $P(\restr u F) = 0$. Donc $\restr u F$ est diagonalisable.
\end{proof}


\begin{theorem}
	$u \in \LE$ est diagonalisable \ssi $\chi_u$ est scindé et la dimension de chaque sous-espace vectoriel est égale à la multiplicité
	algébrique de la valeur propre associée.
\end{theorem}

\begin{proof}
	$u \text{ diagonalisable } \implies E = \bigoplus_{\lambda \in \spec(u)} \ker(u - \lambda \id_E)$\\
	et donc $\chi_u = \prod_{\lambda \in \spec(u)} (X - \lambda)^{\dim \ker(u - \lambda \id_E)}$\\
	$E = \bigoplus_{\lambda \in \spec(u)} \ker(u - \lambda \id_E)^{m_a(\lambda)}$.

	%TODO
\end{proof}


\begin{coro}
	$u \in \LE$ est diagonalisable \ssi $\sum_{\lambda \in \spec(u)} \dim \ker(u - \lambda \id_E) = \dim E$.
\end{coro}

\begin{proof}
	La dimension des sous-espaces propres $\leq$ la dimension du sous-espace caractéristique. %TODO add stuff.
\end{proof}
