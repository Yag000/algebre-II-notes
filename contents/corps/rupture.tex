\subsection{Corps de rupture et corps de décomposition}

Étant donnés $K$ un corps et $P\in K[X]$, on cherche une extension $L$ de $K$ telle que $P$ ait une racine dans $L$ ou toutes ses racines dans $L$.



\begin{definition}
	Soit $P$ un polynôme irréductible dans $K[X]$.\\
	On dit qu'une extension $L$ de $K$ est un corps de rupture de $P$ sur $K$ s'il existe une
	racine $\alpha \in L$ de $P$ telle que $L = K[\alpha] = K(\alpha)$. \\
	Ainsi, un corps de rupture est une extension dans laquelle $P$ a une racine et
	qui est minimale pour cette propriété.
\end{definition}

\begin{theorem}
	Pour tout polynôme irréductible $P$ de $K[X]$ il existe un corps de rupture $L$.
	De plus, $L$ est unique à un $K$-isomorphisme près.
\end{theorem}


\begin{remarque}
	L'unicité n'est pas vraie si $P$ est non irréductible.
	$$ P(X) = (X^2 - 2) (X^2 - 3) \in \Q[X]$$
	$\Q(\sqrt{2})$ et $\Q(\sqrt{3})$ ne sont pas $\Q$ isomorphes car
	$\Q(\sqrt{3})$ ne contient pas les racines de $X^2-2$.\\
	En effet, $\Q(\sqrt{3}) = \left\{  a + b\sqrt{3} \mid a,b \in \Q  \right\}$
	$$ (a + b\sqrt{2}) ^2 = a^2 + 2ab\sqrt{3}  + 3b \neq 2 \text{ sauf si } ab = 0$$
	Si $b=0, \ a = \sqrt{2} = \frac{p}{q}$ ce qui n'est pas possible. \\
	Si $a=0, \ b\sqrt{3} = \sqrt{2} \implies \sqrt{6} \in \Q $ ce qui n'est pas possible non plus.
\end{remarque}

\begin{proof}

	\begin{itemize}
		\item Comme $P$ est irréductible sur $K$ on a :
		      $$ L = K[X]/<P> \text{ est un corps car } K[X] \text{est principal}$$
		      $L$ est une extension de $K$ donc le plongement:
		      \begin{eqnarray*}
			      \phi: K &\to& L\\
			      \lambda   &\mapsto &\bar{\lambda}
		      \end{eqnarray*}
		      est un morphisme de corps.\\
		      Si on note $\alpha = \bar{X}$ la classe de $X$ dans $L$. $P(\alpha) = 0, L = K[\alpha]$
		      donc $L$ est un corps de rupture de $P$ sur $K$.
		\item Si $L$ est un corps de rupture de $P$ dans $K$ \\
		      Soit $\alpha'$ tel que $L' = K[\alpha']$ et $P(\alpha') = 0$, alors l'application
		      \begin{eqnarray*}
			      \phi : K[X] &\to& L'\\
			      Q &   \mapsto & Q(\alpha')
		      \end{eqnarray*}
		      et surjective de noyau $<P> (\ker \phi \supset <P>)$\\
		      Par un théorème de factorisation, on obtient un isomorphisme entre $L$ et $L'$ qui est un $K$-isomorphisme.
	\end{itemize}
\end{proof}

\begin{example}
	$\C$ est le corps de rupture de $X^2+1$ dans $\R$.\\
	$\Q(i)$ est le corps de rupture de $X^2+1$ dans $\Q$.\\
	$\Q(\sqrt[3]{2})$ est le corps de rupture de $X^3-2$ dans $\Q$.\\
	Mais c'est aussi le cas de $\Q(j\sqrt[3]{2})$ et $\Q(j^2\sqrt[3]{2})$, avec $j = e^{\frac{2i\pi}{3}}$.
	De plus, $\Q(\sqrt[3]{2}) \subset \R$, ce qui n'est pas le cas des autres.

	Lorsque $P$ est de degré 2, il n'y a qu'un seul corps de rupture. Si $\alpha$ est une racine de $P = X^2 + aX + b$,
	alors $-a-\alpha$ est l'autre racine.
\end{example}

\begin{remarque}[Important]
	Si $L$ est un corps de rupture pour $P$ dans $K$, alors:
	$$ [L : K] = \deg P$$
	$P(\alpha) = 0$, et donc $\set {1, \alpha, \alpha^2, \ldots, \alpha^{\deg P -1}}$ est une base de $L$ sur $K$ (génératrice grâce à la division euclidienne).
\end{remarque}


\begin{definition}
	Soient $K$ un corps et $P\in K[X]$ (on ne suppose pas forcément $P$ irréductible).\\
	On dit qu'une extension $L$ de $K$ est un corps de décomposition pour $P$ sur $K$ si et seulement si $L$ vérifie les propriétés suivantes:
	\begin{itemize}
		\item $P$ est scindé sur $L$ (produit de polynômes de degré 1).
		\item $L$ est engendré comme corps (ou comme anneau) par les racines de $P$ sur $K$.
	\end{itemize}

	Ainsi un corps est décomposition est une extension minimale de $K$ pour laquelle $P$ est scindé.
\end{definition}

\begin{theorem}
	Pour tout $P\in K[X]$, il existe un corps de décomposition pour $P$ sur $K$ qui est unique à $K$-isomorphisme près.
\end{theorem}

\begin{proof}
	\begin{itemize}
		\item Existence:\\
		      On procède par récurrence sur le degré de $P$.
		      \begin{itemize}
			      \item Pour $\deg P \leq 1$, c'est évident: $L = K$
			      \item Soit $Q$ un facteur irréductible sur $K$ de $P$.
			            Comme $Q$ est irréductible, il admet un corps de rupture sur $K, \ K' = [x], \ Q(x) = 0$.\\
			            On écrit $P(X) = (X-x)P_1(X)$, avec $P_1 \in \K'[X]$. On a $\deg P_1 = \deg P -1$.\\
			            On peut alors appliquer l'hypothèse de récurrence à $P_1$ sur $K'$.\\
			            Il existe $L = K(x_2, \cdots,  x_n)$ tel que $P_1$ est scindé sur $L$ et
			            $x_2, \cdots, x_n$ sont des racines de $P$. On a donc $L = K(x, x_2, \cdots, x_n)$ est bien un
			            corps de décomposition pour $P$ sur $K$:
		      \end{itemize}
		\item Unicité:\\
		      On démontre par recurrence sur le degré de $P$ l'énoncé suivant:\\
		      "Si $\phi : K \to K'$ est un isomorphisme de corps et $P$ un polynôme de $K[X]$ et $L$ et $L'$
		      sont des corps de décomposition pour $P$ (respectivement pour $\phi(P)$) sur $K$ (respectivement sur $K'$),
		      alors il existe un morphisme de corps $\psi: L \to L'$ qui prolonge $\phi$"
		      On obtiendra le résultat voulu en prenant $\phi = id_K$.\\

		      Si $P$ est scindé, on a $L=K, L'=K'$ et l'affirmation est évidente.\\

		      Sinon, on considère $\alpha$ une racine de $P$ dans $L\K$ de polynôme minimal $Q$ (on sait que $Q \mid P$).\\
		      $\phi (Q)$ admet une racine $\alpha' \in L'$ et $K[\alpha]$ et $K'[\alpha']$ sont des corps de rupture sur $Q$(respectivement $\phi(Q)$)
		      sur $K$ (respectivement sur $K'$).\\

		      On prolonge $\phi$ en $\phi_1 : K[\alpha] \to K'[\alpha']$ en posant $\phi_1(\alpha) = \alpha'$.\\

		      Soit $R\in K[X]$, $\phi_1(R(\alpha)) = \phi(R)(\alpha')$ ce qui a bien un sens puisque les polynômes minimaux de $\alpha$ et
		      $\alpha'$ sur $K$ et $K'$ sont $Q$ et $\phi(Q)$ respectivement.\\

		      On a $P(X) = (X-\alpha)P_1$ et $\phi(P)(X) = (X-\alpha'\phi_1(P_1)$, on a juste à appliquer l'hypothèse de récurrence à $P_1$ et $\phi_1(P_1)$
		      sur $K[\alpha]$ et $K'[\alpha']$.\\
		      On obtient bien $\psi : L \to L'$ qui prolonge $\phi_1$ et donc $\phi$.
	\end{itemize}
\end{proof}


\begin{remarque}
	L'unicité est "meilleure" que pour les corps de rupture.\\
	Si $L$ et $L'$ sont deux corps de décomposition pour $P$ sur $K$,et que
	$l \subset M$ et $l' \subset M$, avec $M/K$ une extension de corps, alors
	$$L = L' = K(x_1, \ldots, x_n)$$
	où $x_1, \ldots, x_n$ sont les racines de $P$ dans $M$.
\end{remarque}
