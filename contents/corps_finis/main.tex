
\section{Corps finis}

Un corps fini est un corps qui a un nombre fini d'éléments.

Sa caractéristique est forcément $p$ premier.
Si c'était 0, alors $\Z$ s'injecterait dans $K$ et $\Z$ serait infini.

$K$ peut être vu comme une extension de $\F_p$ via le morphisme:
\begin{eqnarray*}
	\F_p & \to & K \\
	\bar{1} &\mapsto& 1_K
\end{eqnarray*}

$K$ est en particulier un espace vectoriel de dimension finie ${K:\F_p}$ et $|K| = p ^{[K:\F_P]} $


Ici $\F_p = \left( \Zn{p}, +, \times \right)$, $p$ premier.
A isomorphisme près, il y a un seul corps à $p$ éléments.

$\F_p = \left( \Zn, +, \times \right)$ est un corps \ssi $n$ est premier.


\begin{theorem}
	Soit $q = p^n$, avec $p$ premier et $n \geq 1$. Alors il existe un corps de cardinal $q$ unique à isomorphisme près.
	C'est le corps de décomposition pour $X^q-X$ sur $\F_p$. On le note $\F_q$.
\end{theorem}


\begin{rappel} [Utile pour la preuve]
	Si $\alpha$ annule $P$ et sa multiplicité est supérieure ou égale à 2, alors $P'(\alpha) = 0$:
	$$P(X) = (X-\alpha)^2Q(X) \implies P'(X) = 2(X-\alpha)Q(X) + (X-\alpha)^2Q'(X) = 0$$
\end{rappel}


\begin{proof}
	\begin{itemize}
		\item Existence:\\
		      Soit $K$ le corps de décomposition de $X^q-C$ sur $F_p$. \\
		      On note $K'$ l'ensemble des racines dans $K$ de $X^q-X$. \\
		      $K'$ est en fait un corps: $0, 1 \in K'$. Si $x, y \in K'$'.\\
		      Montrons que $(x+y)^q = x^q + y^q = x+ y$.\\
		      Pour montrer l'identité précédente, on observe que pour tout $>0$ et pour tout $x,y \in K$ on a
		      $(x+y)^{p^k}= x^{p^k} + y^{p^k}$. Cela se montre par récurrence grâce à la formule $(x+y)^p= x^p + y^p$.\\
		      $p \mid \binom{p}{k}$ lorsque $1 \leq k \leq p-1$ aussi $(x+y)^p = \sum^p_{k=0} \binom{p}{k} x^k y^{p-k} = x^p + y^p$\\

		      $x\in K'$, alors $-x$ est aussi dans $K'$.
		      En effet, si $car K = 2$, $-x = x -2x = x$. \\
		      Si $car K$ est impaire, $(-x)^q = -x^q = -x$. \\
		      Évidemment $x,y \in K' \implies xy \in K'$, car $(x*y)^q = x^qy^q = xy$\\
		      $x\in K' x \neq 0 \implies \frac{1}{x} \in K'$.

		      Alors $K'$ est un sous corps de $K$.\\
		      Donc par la définition de corps de décomposition $K' = K$.\\

		      La dérivée de $X^q-X$ est $qX^{q-1} - X$ et $qX^{q-1} - 1 = 1$\\
		      Donc toutes les racines du polynôme sont simples et il y en a exactement $q$. $K$ est bien un corps de cardinal $q$.

		\item Unicité:\\
		      On considère $L$ un corps de cardinal $q$.\\
		      On sait que $\forall x \in L \setminus \{0\},\  x^{q-1} = 1$ (théorème de Lagrange appliqué au groupe multiplicatif $L\setminus \set 0$ et
		      cardinal $q-1$), donc:
		      $$\forall x \in L, \ x^q -x = 0$$
		      $X^q-X$ est scindé dans $L$ (car il a $q$ racines distinctes).\\
		      $L$ contient un corps de décomposition pour $X^q-X$ sur $\F_p$. Autrement dit, un $K_1$ qui est
		      isomorphe à $K$ (d'après les propriétés de décomposition).\\
		      $L_1 \subset L$ et $|K_1| = q = |L| \implies K_1 = L \cong K$.
	\end{itemize}
\end{proof}

\begin{exemple}

	\begin{itemize}
		\item $P = X^2+X+1$ est irréductible sur $\F_2$ et
		      $$\F_2[X]/ <x^2+x+1> \cong F_4$$

		      On prend $\alpha$ une racine de $x^2+x+1$.
		      $$\F_2[X]/ <x^2+x+1> = \set{ a + \alpha b \mid a,b \in \F_2} \cong \F_4$$
		      La table de multiplication:

		      \begin{center}
			      \begin{tabular}{c|c|c|c|c}
				      $*$        & 0          & 1        & $\alpha$    & $1+\alpha$  \\
				      \hline
				      0          & 0          & 1        & $\alpha$    & $1+\alpha$  \\
				      \hline
				      1          & 1          & 1        & $\alpha$    & $\alpha +1$ \\
				      \hline
				      $\alpha$   & $\alpha$   & $\alpha$ & $\alpha +1$ & 1           \\
				      \hline
				      $1+\alpha$ & $1+\alpha$ & $\alpha$ & 1           & $\alpha$    \\
			      \end{tabular}
		      \end{center}

		      Car on a $\car \ \F_4 = 2$ et donc $\alpha^2 = (\alpha^2 + \alpha + 1 ) + \alpha + 1 =  P(\alpha) + \alpha + 1 = \alpha + 1$. \\
		      De plus $\alpha(\alpha + 1) = \alpha^2 + \alpha = \alpha + 1 + \alpha = 1$.
		\item$x^3+x+1$ est irréductible sur $\F_2$ et $\F_2[X]/ <x^3+x+1> \cong F_6$.
		\item$ \F_3[X]/ <x^2+1> \cong F_9$
	\end{itemize}
\end{exemple}


\begin{exercice}
	Si $K$ est un corps fini et $P \in K[X]$ irréductible sur $K$, alors le
	corps de rupture de $P$ sur $K$ est aussi un corps de décomposition pour $P$ sur $K$.
\end{exercice}

\begin{remarque}
	$\F_{p^n}$ est une extension de $\F_{p^m}$ \ssi $m$ divise $n$.\\
	Ainsi $\F_8$ n'est pas une extension de $\F_4$. \\
\end{remarque}

\begin{proof}
	\begin{itemize}
		\item $\degExt {F_{p^n}}{F_{p^m}} = d \in \N$\\
		      $p^n = |F_{p^n}| = |F_{p^m}|^d = p^{md} \implies n = md$.

		\item Réciproquement, si $m \mid n$, on écrit $n = md$\\
		      On a $X^{p^m - 1} - 1$ divise $X^{p^n-1} - 1$  et $X^{p^m} - X$ divise $X^{p^n} - X$\\
		      Toutes les racines de $X^{p^m} - X$ sont racines de $X^{p^n} - X$, donc $F_{p^m} \subset F_{p^n}$.
	\end{itemize}
\end{proof}



\begin{remarque}[Morphisme de Frobenius]
	Soit $K$ un corps de caractéristique $p >0$ et $\psi : K \to K$ définie par $\psi(x) = x^p$.\\
	Alors $\psi$ est un morphisme de corps, donc injectif. Si $K$ est fini, alors $\psi$ est un automorphisme de corps.\\
	Ainsi $\psi(x + y) = (x+y)^p = x^p + y^p, \forall x,y \in K$.\\
	De plus $\set {x \in K \mid x^p = x} = \F_p$.\\
	Cela fournit une famille de morphismes $K \to K$:
	$$ \psi,\ \underbrace{\psi \circ \cdots \circ \psi}_{n \text{ fois}} = \psi^n : \begin{cases}
			K \to K \\
			x \mapsto x^{p^n}
		\end{cases}$$

	Pour $P \in F_q[X], \ q = p^n$, on a $\psi(P(x)) = P (\psi(x))$, car $F_q$ est de caractéristique $p$.
\end{remarque}


\subsection{Polynômes irreductibles sur un corps fini}


Soit $I(n,q)$ le plus petit polynôme unitaire de degré $n$ irréductible sur $\F_q$ dans $\F_q[X]$.

\begin{theorem}
	Pour tout $q$ puissance d'un nombre premier et $n \geq 1$ on a $I(n,q) > 0$.\\

	Plus précisément:
	$$ I(n,q) = \frac{1}{n} \sum_{d|n} \mu(\frac{n}{d}) q^d$$

	où $\mu$ est la fonction de Möbius:

	\[ \mu(k) = \left\{\begin{array}{ll}
			0      & \text{si il existe } l \text{ premier tel que } l^2 \text{ divise } k                            \\
			(-1)^r & \text{si } k = p_1 \cdot p_2 \cdot \ldots \cdot p_r \text{ avec } p_i \text{ premiers distincts}
		\end{array} \right.\]

\end{theorem}

\begin{lemma}
	$$\sum_{d|n} \mu(d) = \left\{\begin{array}{ll}
			1 & \text{si } n = 1 \\
			0 & \text{sinon}
		\end{array} \right.$$
\end{lemma}

\begin{proof}
	On peut se ramener à $n$ sous facteur carré $n = p_1 \ldots p_r$ avec $p_i$ premiers distincts.\\
	\begin{eqnarray*}
		\sum_{d|n} \mu(d) &=& \sum_{j=0}^r (-1)^j \#\set{d \mid n : \text{nombre de facteurs premiers de } d \text { est } j} \\
		&=& \sum_{j=0}^r (-1)^j \binom{r}{j} \\
		&=& (1-1)^r = 0 \text{ si } r > 1
	\end{eqnarray*}
\end{proof}

\begin{lemma}
	Soient $f$ et $g$ des applications de $\N^*$ dans $\C$, alors:
	$$ \forall n \in \N^*, g(n) = \sum_{d|n} f(d) \iff \forall n \in \N^*, f(n) = \sum_{d|n} \mu(\frac n d) g(d) $$

\end{lemma}

\begin{proof}
	Si $g(n) = \sum_{d|n} f(d)$ alors:
	\begin{eqnarray*}
		\sum_{d|n} \mu(\frac n d) g(d) &=& \sum_{d|n} \mu(\frac n d) \sum_{l|d} f(l) \\
		&=& \sum_{l|n} f(b) \underbrace{\sum_{d|n,\ l|d}}_{ d = ld', \ d' | \frac n l} \mu(\frac n d) \\
		&=& \sum_{l | n} f(l) \sum_{d' | \frac n l} \mu(\frac n {ld'}) \\
		&=& \sum_{l | n} f(l) \underbrace{\sum_{m | \frac n l} \mu(m)}_{ 0, \text{ sauf si } l = n } \\
		&=& f(n)
	\end{eqnarray*}
\end{proof}

\begin{proof}[du théorème]
	Il suffit de montrer que $q^n = \sum_{d|n} d I(d,q)$.\\

	Considérons $Q = X^{q^n} - X$ et sa décomposition en produits de polynômes irréductibles unitaires:
	$$ Q = P_1 \ldots P_r$$

	Tous les $p_j$ sont distincts car si $P_i = P_j (i \neq j)$.\\
	$$P_j^2 | Q \implies P_j | Q' = - 1$$
	Montrons que les $P_j$ sont exactement les polynômes irréductibles unitaires dont le degré divise $n$. Soit $P$ un
	polynôme irréductible de degré $d$. Supposons $d | n$. Soit $x$ dans un corps de rupture pour $P$ tel que $P(x) = 0$.
	Ce cors est de degré $d$ sur $\F_q$. Il est donc isomorphe à $\F_{q^d}$. Donc $x^{q^d} = x$.\\
	Avec $d|n$, on a $x^{q^n} = x$. Soit $m = kd$. Si $\psi_{q^d}$ est le morphisme de Frobenius $x \mapsto x^{q^d}$, alors
	$$\psi_{q^d}^k(x) = \psi_{q^d} \circ \dots \circ \psi_{q^d}(x) = x^{q^n} = x$$
	On a $Q(x) = 0$. Comme $P$ est le polynôme minimal de $x$ sur $\F_q$, on a donc $P | Q$.\\

	Réciproquement, si $P | Q$ de degré $d$.\\
	Toute racine $x$ de $P$ anule $Q$, donc $\F_{q^d}$ est le corps de décomposition de $\Q$ qui contient $\F_{q^d} = F_q[x]$, donc
	$d | n$ (comme pour le théorème de la base télescopique $\F_q \subset \F_{q^d} \subset \F_{q^n}$).\\

	En calculant le degré de $Q$ on en déduit que
	$$q^n = \sum_i \deg P_i = \sum_{d|n} d I(d,q)$$
\end{proof}


\begin{example}
	$ q = 2, \ n = 4, \ 2^4 = 16 $
	$$	X ^16 - X = X(X-1)(X^2 + X + 1)(X^4 + X + 1)(X^4 + X^3 + 1)(X^4 + x^3 + x^2 + x + 1) $$
	$$	I(2,2) = \frac{1}{2}\sum_{d|2} \mu(\frac{2}{d}) 2^d = \frac{1}{2}(4-2) = 1$$
	$$	I(4,2) = \frac{1}{4}\sum_{d|4} \mu(\frac{4}{d}) 2^d = \frac{1}{4}(2^4 - 2^2) = 3$$
\end{example}



\begin{coro}
	Pour tout $n$ et tout $q$ puissance d'un nombre premier:
	$$ I (n, q) > 0 $$
\end{coro}


\begin{proof}
	\begin{eqnarray*}
		n I(n, q) & = & \sum_{d|n} \mu(\frac{n}{d}) q^d \\
		&>& q^n - \sum_{d|n,\ d < n} q^d \\
		&>& q^n - \sum_{1 \leq d\leq \lfloor \frac n 2 \rfloor} q^d \\
		&=& q^n - q\frac{q^{\lfloor \frac n 2 \rfloor} - 1}{q-1} \\
		&>& 0
	\end{eqnarray*}
\end{proof}


\subsection{Critères de réductibilité sur $\Q$ et $\F_p = \Zn p$}
Pour étudier la réductibilité d'un polynôme sur $\Q$ on peut toujours se ramener à un
polynôme sur $\Z$ primitif.



\begin{prop}
	Soit $P = a_nX^n + \ldots + a_0 \in \Z[X]$.\\
	Soit $p$ un nombre premier tel que $p \nmid a_n$.\\
	Si $\bar{P}$,la réduction modulo $p$, est irréductible sur $\F_p$, alors $P$ est irréductible sur $\Q$.
	De plus, si $P$ est primitif, alors $P$ est irréductible sur $\Z$.
\end{prop}


\begin{remarque}
	$a \nmid b$ est essentiel : $2X^2 - X + 1$ est irréductible sur $\F_2$ mais réductible sur $\Q$, 1 est une racine.
\end{remarque}

\begin{remarque}
	Ce critère est une condition suffisante mais pas nécessaire.\\
\end{remarque}


\begin{proof}
	On suppose $P$ primitif, réductible sur $\Q$ et donc aussi sur $\Z$.\\
	Alors il existe $R, S \in \Q[X]$ , tels que $P = RS$
	$$ \exists a, b \in \N* \text{ tels que } aR, bS \in \Z[X]$$
	$$ abP = aRbS$$
	$$ c(abP) = ab = c(aR)c(vS) $$
	$$P =   \frac{c(aR)}{c(aR)} \frac{c(bS)}{c(bS)}$$
	Modulo $p$ $\bar{P} = \bar{Q}\bar{R}$.
	De plus, on a:
	\begin{itemize}
		\item $\deg Q = \deg \bar{Q}$
		\item $\deg R = \deg \bar{R}$

	\end{itemize}
	car leurs coefficients dominants ne divisent pas $p$.\\
	Donc $\bar{P} = \bar{Q}\bar{R} \implies \bar{P}$ n'est pas irréductible dans $\Zn{p}[X]$.\\
	On a donc montré la contraposée de la proposition.
\end{proof}


\begin{prop}
	Soit $P \in K[X]$ et $\deg P = n$, Si $P$ n'a pas de racines dans toute extension
	de $K$ de degré au plus $\frac{n}{2}$, alors $P$ est irréductible sur $K$.
\end{prop}

\begin{remarque}
	Si $n = 2$ ou $n = 3$ ce résultat dit que si $P$ n'a pas de racines dans $K$, alors $P$ est irréductible.
\end{remarque}


\begin{proof}
	Soit $P$ réductible sur $K$. Alors il existe $Q$ irréductible de degré $\leq \frac{n}{2}$ qui divise $P$.\\
	Soit un corps de rupture pour $Q$ sur $K$.\\
	$$\degExt L K = \deg Q \leq \frac{n}{2}$$
	$L$ contient une racine de $Q$ et donc une racine de $P$.
\end{proof}



\begin{example}
	$P(X) = X^4 + 1 \in \F_p[X]$ avec $p$ premier n'est pas irréductible sur $\F_p$ alors que $X^4 + 1$ est irréductible sur $\Q$.

	Montrons que $P$ a toujours une racine dans $\F_{p^2}$ :
	$F_{p^2}$ est le corps de décomposition de $X^{p^2} - X = X (X^{p^2-1} - 1)$ sur $\F_p$.
	Donc les éléments non nuls de $F_{p^2}$ sont les racines de $X^{p^2-1} - 1$.\\
	$p$ est impair, donc $p^2 - 1 \equiv 0 \mod 8$ car $1,3,5$ et $7$ au carré donnent $1$ modulo $8$.\\

	On choisit $x$ une racine de $X^{p^2-1} - 1$ d'ordre $8$ mais qui ne soit pas d'ordre $4$.

	$x^8-1 = 0 = (X^4-1)(X^4+1)$, donc $x^4 + 1 = 0$.

	Donc $P$ a une racine dans $\F_{p^2}$ et donc $P$ est réductible sur $\F_p$.
\end{example}




