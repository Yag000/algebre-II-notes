\subsection{Polynômes irréductibles sur un corps fini}


Soit $I(n,q)$ le plus petit polynôme unitaire de degré $n$ irréductible sur $\F_q$.

\begin{theorem}
	Pour tout $q$ puissance d'un nombre premier et $n \geq 1$ on a $I(n,q) > 0$.\\

	Plus précisément:
	$$ I(n,q) = \frac{1}{n} \sum_{d|n} \mu(\frac{n}{d}) q^d$$

	où $\mu$ est la fonction de Möbius:

	\[ \mu(k) = \left\{\begin{array}{ll}
			0      & \text{s'il existe } l \text{ premier tel que } l^2 \text{ divise } k                            \\
			(-1)^r & \text{si } k = p_1 \cdot p_2 \cdot \ldots \cdot p_r \text{ avec } p_i \text{ premiers distincts}
		\end{array} \right.\]

\end{theorem}

\begin{lemma}
	$$\sum_{d|n} \mu(d) = \left\{\begin{array}{ll}
			1 & \text{si } n = 1 \\
			0 & \text{sinon}
		\end{array} \right.$$
\end{lemma}

\begin{proof}
	On peut se ramener à $n$ sous facteur carré $n = p_1 \ldots p_r$ avec $p_i$ premiers distincts.\\
	\begin{eqnarray*}
		\sum_{d|n} \mu(d) &=& \sum_{j=0}^r (-1)^j \#\set{d \mid n : \text{nombre de facteurs premiers de } d \text { est } j} \\
		&=& \sum_{j=0}^r (-1)^j \binom{r}{j} \\
		&=& (1-1)^r = 0 \text{ si } r > 1
	\end{eqnarray*}
\end{proof}

\begin{lemma}
	Soient $f$ et $g$ des applications de $\N^*$ dans $\C$, alors:
	$$ \forall n \in \N^*, g(n) = \sum_{d|n} f(d) \iff \forall n \in \N^*, f(n) = \sum_{d|n} \mu(\frac n d) g(d) $$

\end{lemma}

\begin{proof}
	Si $g(n) = \sum_{d|n} f(d)$ alors:
	\begin{eqnarray*}
		\sum_{d|n} \mu(\frac n d) g(d) &=& \sum_{d|n} \mu(\frac n d) \sum_{l|d} f(l) \\
		&=& \sum_{l|n} f(b) \underbrace{\sum_{d|n,\ l|d}}_{ d = ld', \ d' | \frac n l} \mu(\frac n d) \\
		&=& \sum_{l | n} f(l) \sum_{d' | \frac n l} \mu(\frac n {ld'}) \\
		&=& \sum_{l | n} f(l) \underbrace{\sum_{m | \frac n l} \mu(m)}_{ 0, \text{ sauf si } l = n } \\
		&=& f(n)
	\end{eqnarray*}
\end{proof}

\begin{proof}[Démonstration du théorème]
	Il suffit de montrer que $q^n = \sum_{d|n} d I(d,q)$.\\

	Considérons $Q = X^{q^n} - X$ et sa décomposition en produit de polynômes irréductibles unitaires:
	$$ Q = P_1 \ldots P_r$$

	Tous les $p_j$ sont distincts car si $P_i = P_j (i \neq j)$.\\
	$$P_j^2 | Q \implies P_j | Q' = - 1$$
	Montrons que les $P_j$ sont exactement les polynômes irréductibles unitaires dont le degré divise $n$. Soit $P$ un
	polynôme irréductible de degré $d$. Supposons $d | n$. Soit $x$ dans un corps de rupture pour $P$ tel que $P(x) = 0$.
	Ce corps est de degré $d$ sur $\F_q$. Il est donc isomorphe à $\F_{q^d}$. Donc $x^{q^d} = x$.\\
	Avec $d|n$, on a $x^{q^n} = x$. Soit $m = kd$. Si $\psi_{q^d}$ est le morphisme de Frobenius $x \mapsto x^{q^d}$, alors
	$$\psi_{q^d}^k(x) = \psi_{q^d} \circ \dots \circ \psi_{q^d}(x) = x^{q^n} = x$$
	On a $Q(x) = 0$. Comme $P$ est le polynôme minimal de $x$ sur $\F_q$, on a donc $P | Q$.\\

	Réciproquement, si $P | Q$ de degré $d$.\\
	Toute racine $x$ de $P$ annule $Q$, donc $\F_{q^d}$ est le corps de décomposition de $\Q$ qui contient $\F_{q^d} = F_q[x]$, donc
	$d | n$ (comme pour le théorème de la base télescopique, $\F_q \subset \F_{q^d} \subset \F_{q^n}$).\\

	En calculant le degré de $Q$ on en déduit que
	$$q^n = \sum_i \deg P_i = \sum_{d|n} d I(d,q)$$
\end{proof}


\begin{example}
	$ q = 2, \ n = 4, \ 2^4 = 16 $
	$$	X ^16 - X = X(X-1)(X^2 + X + 1)(X^4 + X + 1)(X^4 + X^3 + 1)(X^4 + x^3 + x^2 + x + 1) $$
	$$	I(2,2) = \frac{1}{2}\sum_{d|2} \mu(\frac{2}{d}) 2^d = \frac{1}{2}(4-2) = 1$$
	$$	I(4,2) = \frac{1}{4}\sum_{d|4} \mu(\frac{4}{d}) 2^d = \frac{1}{4}(2^4 - 2^2) = 3$$
\end{example}



\begin{coro}
	Pour tout $n$ et tout $q$ puissance d'un nombre premier:
	$$ I (n, q) > 0 $$
\end{coro}


\begin{proof}
	\begin{eqnarray*}
		n I(n, q) & = & \sum_{d|n} \mu(\frac{n}{d}) q^d \\
		&>& q^n - \sum_{d|n,\ d < n} q^d \\
		&>& q^n - \sum_{1 \leq d\leq \lfloor \frac n 2 \rfloor} q^d \\
		&=& q^n - q\frac{q^{\lfloor \frac n 2 \rfloor} - 1}{q-1} \\
		&>& 0
	\end{eqnarray*}
\end{proof}

