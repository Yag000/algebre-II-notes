\subsection{Critères de réductibilité sur $\Q$ et $\F_p = \Zn p$}
Pour étudier la réductibilité d'un polynôme sur $\Q$ on peut toujours se ramener à un
polynôme sur $\Z$ primitif.



\begin{prop}
	Soit $P = a_nX^n + \ldots + a_0 \in \Z[X]$.\\
	Soir $p$ un nombre premier tel que $p \nmid a_n$.\\
	Si $\bar{P}$,la réduction modulo $p$, est irréductible sur $\F_p$, alors $P$ est irréductible sur $\Q$.
	De plus, si $P$ est primitif, alors $P$ est irréductible sur $\Z$.
\end{prop}


\begin{remarque}
	$a \nmid b$ est essentiel : $2X^2 - X + 1$ est irréductible sur $\F_2$ mais réductible sur $\Q$, 1 est une racine.
\end{remarque}

\begin{remarque}
	Ce critère est une condition suffisante mais pas nécessaire.\\
\end{remarque}


\begin{proof}
	On suppose $P$ primitif, réductible sur $\Q$ et donc aussi sur $\Z$.\\
	Alors il existe $R, S \in \Q[X]$ , tels que $P = RS$
	$$ \exists a, b \in \N* \text{ tels que } aR, bS \in \Z[X]$$
	$$ abP = aRbS$$
	$$ c(abP) = ab = c(aR)c(vS) $$
	$$P =   \frac{c(aR)}{c(aR)} \frac{c(bS)}{c(bS)}$$
	Modulo $p$ $\bar{P} = \bar{Q}\bar{R}$.
	De plus, on a:
	\begin{itemize}
		\item $\deg Q = \deg \bar{Q}$
		\item $\deg R = \deg \bar{R}$

	\end{itemize}
	car leurs coefficients dominants ne divisent pas $p$.\\
	Donc $\bar{P} = \bar{Q}\bar{R} \implies \bar{P}$ n'est pas irréductible dans $\Zn{p}[X]$.\\
	On a donc montré la contraposée de la proposition.
\end{proof}


\begin{prop}
	Soit $P \in K[X]$ et $\deg P = n$ n'a pas de racines dans toute extension
	de $K$ de degré au plus $\frac{n}{2}$, alors $P$ est irréductible sur $K$.
\end{prop}

\begin{remarque}
	Si $n = 2$ ou $n = 3$ ce résultat dit que si $P$ n'a pas de racines dans $K$, alors $P$ est irréductible.
\end{remarque}


\begin{proof}
	Soit $P$ réductible sur $K$. Alors il existe $Q$ irréductible de degré $\leq \frac{n}{2}$ qui divise $P$.\\
	Soit un corps de rupture pour $Q$ sur $K$.\\
	$$\degExt L K = \deg Q \leq \frac{n}{2}$$
	$L$ contient une racine de $Q$ et donc une racine de $P$.
\end{proof}



\begin{example}
	$P(X) = X^4 + 1 \in \F_p[X]$ avec $p$ premier n'est pas irréductible sur $\F_p$ alors que $X^4 + 1$ est irréductible sur $\Q$.

	Montrons que $P$ a toujours une racine dans $\F_{p^2}$ :
	$F_{p^2}$ est le corps de décomposition de $X^{p^2} - X = X (X^{p^2-1} - 1)$ sur $\F_p$.
	Donc les éléments non nuls de $F_{p^2}$ sont les racines de $X^{p^2-1} - 1$.\\
	$p$ est impair, donc $p^2 - 1 \equiv 0 \mod 8$ car $1,3,5$ et $7$ au carré donnent $1$ modulo $8$.\\

	On choisit $x$ une racine de $X^{p^2-1} - 1$ d'ordre $8$ mais qui ne soit pas d'ordre $4$.

	$x^8-1 = 0 = (X^4-1)(X^4+1)$, donc $x^4 + 1 = 0$.

	Donc $P$ a une racine dans $\F_{p^2}$ et donc $P$ est réductible sur $\F_p$.
\end{example}



