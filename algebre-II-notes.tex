\documentclass{article}
\usepackage[utf8]{inputenc}
\usepackage[T1]{fontenc}

\usepackage{amsmath}
\usepackage{amssymb} 
\usepackage{amsthm}  
\usepackage{dsfont}
\usepackage{mathrsfs}
\usepackage{mathtools}

\usepackage{tikz-cd}

\usepackage{geometry}

\usepackage{hyperref}        

%% Color links
\usepackage{xcolor}
\hypersetup{
    colorlinks,
    linkcolor={red!50!black},
    citecolor={blue!50!black},
    urlcolor={blue!80!black}
}

\usepackage[french]{babel}

\usepackage[shortlabels]{enumitem}

\usepackage{fancyhdr}

\usepackage{algebre-II-notes}

\fancypagestyle{toc}{%
\fancyhf{}%
\fancyhead[L]{\rightmark}%
\fancyhead[R]{\thepage}%
}

\pagestyle{toc}



\begin{document}
\begin{titlepage}
	\newcommand{\HRule}{\rule{\linewidth}{0.5mm}}
	\center

	\HRule\\[0.4cm]

	\textsc{\Large Algèbre II}\\[0.5cm]
	\textsc{\large Un ensemble compréhensible de notes de cours}\\[0.5cm]

	\HRule\\[1.5cm]


	{\large\textit{Auteur}}\\
	Yago \textsc{Iglesias}


	\vfill\vfill\vfill

	{\large\today}

	\vfill

\end{titlepage}


\tableofcontents


\section{Introduction}

Ce document est un recueil de notes de cours sur l'algèbre niveau L3. Il est
basé sur les cours de M.~\textsc{Régis de la Bretèche} à Université Paris Cité, cependant toute
erreur ou inexactitude est de ma responsabilité.
Si bien \textsc{Yago IGLESIAS} est l'auteur de ce document, il n'est pas
le seul contributeur. En effet, de nombreux étudiants ont participé à la
rédaction de ce document. Leurs noms sont disponibles dans la section
contributeurs du répertoire \href{https://github.com/Yag000/algebre-II-notes/graphs/contributors}{GitHub}.
Un remerciement particulier à \textsc{Erin Le Boulc'h}, \textsc{Gabin Dudillieu} et \textsc{Mathusan Selvakumar} pour leur
participation active à la rédaction de ce document.
\vspace{0.5cm}

Le document est structuré en 3 parties. La première partie est consacrée aux anneaux et aux polynômes.
La deuxième partie est consacrée aux corps et aux extensions de corps. Et la dernière partie porte sur
la réduction des endomorphismes.
\vspace{0.5cm}

Toute erreur signalée ou remarque est la bienvenue.
Sentez-vous libres de contribuer à ce document par le biais de \href{https://github.com/Yag000/algebre-II-notes}{GitHub},
où vous pouvez trouver le code source de ce document et une version pdf à jour.
Si vous n'êtes pas familiers avec \textit{Git} ou \LaTeX, vous pouvez toujours me contacter
par \href{mailto: yago.iglesias.vazquez@gmail.com}{mail}.



\section{Anneaux de Polynômes}

\subsection{Construction formelle}

\begin{definition}[Anneau de polynômes]

	Soit $A$ un anneau commutatif. On note $A^{\mathbb{N}}$ l'ensemble des suites presque nulles d'éléments de $A$.
	On munit $A^{\mathbb{N}}$ d'une structure d'anneau en posant:
	\begin{itemize}
		\item $(a_n)_{n \in \mathbb{N}} + (b_n)_{n \in \mathbb{N}} = (a_n + b_n)_{n \in \mathbb{N}}$
		\item $(a_n)_{n \in \mathbb{N}} \cdot (b_n)_{n \in \mathbb{N}} = (c_n)_{n \in \mathbb{N}}$ où $c_n = \sum_{k=0}^{n} a_k b_{n-k}$
	\end{itemize}

	On note $A[X]$ l'anneau commutatif (proposition à montrer si besoin)  $A^{\mathbb{N}}$.

\end{definition}

\begin{definition}[Anneau intègre]
	Un anneau $(A, +, \cdot)$ est intègre s'il est commutatif, non trivial et pour tout $x, y \in A$,
	\begin{equation*}
		xy = 0 \implies x = 0 \ \text{ ou } \ y = 0
	\end{equation*}
\end{definition}

\begin{prop}
	Soit $A$ un anneau commutatif.
	Soient $P, Q \in A[X]$. Alors $\deg(PQ) \leq \deg(P) + \deg(Q)$.
	De plus, si $A$ est intègre, $\deg(PQ) = \deg(P) + \deg(Q)$.
\end{prop}

\begin{proof}
	Soient $P=a_n x^n+\ldots+a_0$ et $Q=b_m x^m+$ $\cdots+b_0$ avec $a_n \neq 0$ et $b_m \neq 0$.
	Ainsi, $\deg(P)=n$ et $\deg(Q)=m$. Le terme de plus haut degré dans $P Q$ vient de $a_n X^n \cdot b_m X^m=a_n \cdot b_m X^{n+m}$.
	Par conséquent, $\deg(P Q) \leq n+m=\deg(P)+\deg(Q)$.

	Si A est intègre, alors $a_n b_m \neq 0$ si $a_n \neq 0$ et $b_m \neq 0$. Ainsi:
	$$
		\deg(P Q)=n+m=\deg(P)+\deg(Q).
	$$
\end{proof}

\begin{coro}
	$A[X]$ intègre $\iff A$ intègre.
\end{coro}

\begin{proof}
	\begin{itemize}
		\item $\Rightarrow$ C'est immédiat, car un sous-anneau d'un anneau intègre est intègre.\\
		\item $\Leftarrow$ Soient $P, Q \in A[X]$ non nuls. On a donc $\deg(P) \geqslant 0$ et $\deg(Q) \geqslant 0$.\\
		      Mais alors, on a :
		      $$ \deg(P Q)=\deg(P)+\deg(Q) \geqslant 0, \text{par la proposition précédente} $$
		      Ce qui entraîne que $P Q$ est non nul, d'où l'intégrité de $A[X]$.
	\end{itemize}
\end{proof}

\begin{exemple}
	$\F_p=\Zn{p}$, avec $p$ premier, est intègre, donc $\F_p[X] = \Zn{p}[X]$ est intègre.
\end{exemple}

\subsection{Division euclidienne}

\begin{prop}[Division euclidienne]
	Soit $A$ un anneau commutatif. Soient $P, Q \in A[X]$ avec $Q \neq 0$ et
	$Q$ a un coefficient dominant inversible. Alors il existe un unique couple
	$(U, R) \in A[X] \times A[X]$ tel que:
	\begin{itemize}
		\item $P = UQ + R$
		\item $\deg(R) < \deg(Q)$
	\end{itemize}
\end{prop}

\begin{prop}
	Soit $K$ un corps. $K[X]$ est un anneau euclidien et donc principal.
\end{prop}

\begin{proof}
	On commence par noter que $K$ est intègre et donc $K[X]$ aussi.
	Soit $I$ un idéal non nul de $K[X]$. On note $\mathcal{P}$ l'ensemble des degrés des polynômes de $I$:
	\[\mathcal{P} = \{ \deg(P) \mid P \in I \}\]

	$\mathcal{P}$ est non vide et minoré par $0$. Donc il existe $d \in \mathcal{P}$ tel que $d$ est minimal.
	On note $Q$ un polynôme de $I$ de degré $d$. Soit $P \in I$. On réalise la division euclidienne de $P$ par $Q$:
	\[ \exists (U, R) \in K[X] \times K[X] \mid P = UQ + R \text{ et } \deg(R) < \deg(Q) \]
	On a que $R = P - UQ \in I$  car $I$ est un idéal. Comme $\deg(R) < d$ et par définition $d$ est minimal pour
	tous les éléments non nuls, on a que $R = 0$. Donc $P = UQ$ et donc $I = <Q>$.
\end{proof}


\begin{exemple}[Exemple pathologique]
	$\mathbb{Z}[X]$ n'est pas principal. En effet, on peut prendre $I = <2, X>$.
\end{exemple}

\begin{theorem}[Propriété universelle de l'anneau de polynômes]\label{th:prop_univ_anneau_poly}
	Soient $A$ et $B$ des anneaux commutatifs. On considère $f: A \to B$ un morphisme d'anneaux.
	Soit $b \in B$. Alors il existe un unique morphisme d'anneaux
	$\tilde{f}: A[X] \to B$ tel que $\tilde{f}(X) = b$ et $\tilde{f}_{\mid A} = f$.
\end{theorem}

\begin{definition}[Evaluation]
	Soit $A$ un anneau commutatif et $a \in A$. On note $\phi_x: A[X] \to A$ le morphisme d'anneaux
	induit par l'automorphisme trivial de $A$ en utilisant la propriété universelle de l'anneau de polynômes \ref{th:prop_univ_anneau_poly}.

	Ce morphisme correspond à l'évaluation en $x$ : $\phi_x(P) = P(x)$.
\end{definition}

\begin{definition}[Polynôme à plusieurs variables / à n indéterminés]
	Soit $A$ un anneau commutatif. On définit par récurrence sur $n \in \N^*$ l'anneau $A[X_1, \dots, X_n]$:
	\begin{itemize}
		\item $A[X_1] = A[X]$
		\item $A[X_1, \dots, X_n] = A[X_1, \dots, X_{n-1}][X_n]$
	\end{itemize}
\end{definition}


\begin{definition}[Anneau factoriel]
	Soit $A$ un anneau commutatif. On dit que $A$ est factoriel si:
	\begin{itemize}
		\item $A$ est intègre
		\item Tout élément non nul de $A$ est inversible ou est produit d'un nombre fini
		      d'éléments irréductibles ( $a = u p_1 \dots p_n$ avec $u$ inversible et $p_i$ irréductible)
		\item La décomposition est unique à l'ordre près et à l'association près.
	\end{itemize}
\end{definition}

\begin{prop}[Admis]
	Soit $A$ un anneau factoriel. Alors $A[X]$ est factoriel.
\end{prop}


\begin{prop}
	Soit $A$ un anneau intègre tel que tout élément de $A\setminus\{A^{\times}\}$ est produit d'un nombre fini d'éléments irréductibles, alors les assertions suivantes sont équivalentes:
	\begin{enumerate}
		\item $A$ est factoriel
		\item Si $p \in A$ est irréductible, alors l'idéal $<p>$ est premier
		\item Soient $a,b,c \in A\setminus\{0\}$ tels que $a \mid bc$ et $a$ et $b$ sont premiers entre eux. Alors $a \mid c$ (lemme de Gauss).
	\end{enumerate}
\end{prop}

\begin{proof}

	3) $\implies$ 2):\\
	Soit $p \in A$ irréductible. On a que $<p> \neq A$ car $p$ est irréductible, donc pas inversible.
	Si $p$ divise $ab$ et ne divise pas $a$, alors $p$ et $a$ sont premiers entre eux car $p$ est irréductible.
	Donc tout diviseur commun de $p$ et de $c$ est soit inversible soit associé à $p$. Donc $p$ divise $c$.
	Donc d'après 3), $<p>$ est premier.

	\[ ab \in <p>\quad  et \quad a \notin <b> \implies b \in <p> \implies <p> \text{ premier} \]

\end{proof}

\begin{proof}

	2) $\implies$ 1):\\
	Soit $\mathcal{P}$ un système de représentants des irréductibles.
	\begin{equation*}
		u \cdot \prod_{p \in \mathcal{P}} p^{n_q} (\text{divisible par } q^{n_q} \in <q> ) = v \cdot \prod_{p \in \mathcal{P}} p^{m_q} (\text{divisible par } q^{m_q} \in <q>)
	\end{equation*}
	\noindent

	S'il existe $q \in \mathcal{P}$ tel que $m_q > n_q$, alors $q$ divise $u \cdot \prod\limits_{p \ne q} p^{n_q}$ ($\in <q>$), ce qui n'est pas possible par 2).

\end{proof}

\begin{proof}

	1) $\implies$ 3):\\
	$A$ est factoriel. Si $a$  divise $x$, on écrit $a$, $b$, $c$ sous la forme $u \cdot \prod\limits_{p \in \mathcal{P}} p^{v_{p(x)}}$.
	On a alors $\forall p \in \mathcal{P}$, $v_p(a) \leqslant v_p(b) \leqslant v_p(c)$ car $a$ divise $bc$.
	Si $v_p(a) \geqslant b$, alors $v_p(b) = 0$.
	Pour $H_p \in \mathcal{P}$, $v_p(a) \leqslant v_p(c)$, donc $a$ divise $c$, qui vérifie 3).

\end{proof}


\begin{definition}[pgcd]
	Le plus grand commun diviseur est défini ainsi:
	$$d = pgcd (a,b),\  \text{tout diviseur de} \ a  \ \text{et de} \ b \ \text{divise} \ d$$
	et $d$ divise $a$ et $b$
\end{definition}

\begin{remarque}
	Si $K$ est un corps, il y a un seul idéal non nul, qui est $K$ et donc
	tous les $pgcd$ valent $1$.
\end{remarque}

\begin{prop}
	Si $A$ est un anneau factoriel, alors deux éléments non nuls de $A$ admettent un $pgcd$ défini à un facteur inversible près.
\end{prop}

\begin{proof}
	Soit $\mathcal{P}$ un système de représentants des irréductibles de $A$.
	On écrit
	$$ a = u \prod\limits_{p \in \mathcal{P} } p^{n_p} \ \text{où}\  u \in A^\times $$
	$$ b = v \prod \limits_{p \in \mathcal{P} } p^{m_p} \ \text{où}\  v \in A^\times $$
	$$ pgcd(a,b)= \prod \limits_{p \in \mathcal{P}} p^{\min(m_p, m_p)}\  \text{où} \  u \in A^\times $$
	à facteur $\omega \in A^\times$ près
\end{proof}

\begin{exemple}
	$$A  = \mathbb{Z}, \quad  pgcd(-6,2) = 2\ \text{ou}\ -2$$
\end{exemple}

\begin{theorem}[admis]
	$A$ principal $\implies A$ factoriel
\end{theorem}

\begin{exemple}[Anneau factoriel non principal]
	$\Z[X]$ est un anneau factoriel, mais non principal.
\end{exemple}

\begin{prop}
	Dans un anneau principal on écrit
	$$ <a,b> \ = \  <d>, \quad \text{où}\quad d = pgcd (a,b) $$
\end{prop}

\begin{definition}
	Soit $A$ un anneau factoriel, et $P \in A[X]$, le contenu (notée $c(P)$) d'un polynôme $P$ est le
	$pgcd$ de ses coefficients non nuls.
	$P$ est dit primitif si $c(P)=1$ ( ou $c(P) \in A^\times$)
\end{definition}


\begin{exemple}
	$$A  = \mathbb{Z}, \quad  c(3X + 2) = 1$$
	$$A  = \mathbb{Z}, \quad  c(14X^2 + 24X + 2) = 2$$
\end{exemple}


\begin{lemma}[Lemme de Gauss]
	Pour tout $P,Q \in A[X]$ on a
	$$ c(PQ) = c(P)c(Q)$$
	à facteur inversible près.
\end{lemma}

\begin{proof}
	Commençons par montrer que $P$ et $Q$ primitifs implique $PQ$ primitif. \\
	Sinon, il existe un irréductible $p \in A $ tel que $p$ divise tous les coefficients de $PQ$. \\
	Supposons que $P$ et $Q$ sont primitifs. On pose $P = \sum a_iX^i$ et  $Q = \sum b_jX^j$
	On a que $$D = \{i \,\mid \, p\  \text{ne divise pas}\  a_i \}$$ n'est pas vide, car si $D$ est vide alors $ \forall\  i,\, p \,|\, a_i \implies p | c(P)$.
	On note $i_0$ (resp. $j_0$) l'indice minimal tel que $a_{i_0}$ (resp. $b_{j_0}$) ne soit pas divisible par $p$ et :
	\begin{eqnarray*}
		\forall i, \, 0 \leq i \leq i_0\,,& p \mid a_i \\
		\forall j, \, 0 \leq j \leq j_0\,,& p \mid b_j
	\end{eqnarray*}
	On a donc que le coefficient de degré $i_0 + j_0$ de $PQ$ est:
	\begin{eqnarray*}
		PQ_{i_0+j_0}&=& \sum\limits^{i_0 + j_0}_{k=0} a_k b_{i_0 + j_0 - k} \\
		&=& a_{i_0}b_{j_0} + \text{un multiple de } p
	\end{eqnarray*}
	Si $k \neq i_0$, soit  $k\leq i_0 -1$ soit $i_0+j_0 - \leq j_0 -1$ et donc $ p \mid a_kb_{i_0 + j_0 - k}$\\
	Donc le coefficient de degré $i_0+j_0$ de $PQ$ n'est pas divisible par $p$ ce qui contredit les hypothèses.
	Donc on a $P$ et $Q$ primitif $\implies$ $PQ$ primitif. \\
	Dans le cas général
	\begin{eqnarray*}
		c(PQ) &=& c\left(\frac{P}{c(P)}\frac{Q}{c(Q)}c(P)c(Q)\right)\\
		&=& c\left(\frac{P}{c(P))}\frac{Q}{c(Q)}\right)c(P)c(Q)\\
		&=& c(P)c(Q)
	\end{eqnarray*}
	car $\frac{P}{c(P)}$ est un polynôme primitif de $A[X]$ et
	$pgcd(ka, kb) = k pgcd(a,b)$ et donc $c(kP) = kc(P)$.
\end{proof}


\begin{definition}[Corps de fraction]
	Soit $A$ un anneau commutatif intègre.
	On introduit
	$$E = \left\{ (a,b) \in A\times A \mid b \neq 0 \right\}$$
	On munit  $E$ de 2 lois internes:
	\begin{itemize}
		\item $\times : (a,b) \times (a', b') = (aa',bb')$.
		\item $+ : (a,b) + (a', b') = (ab' + ba',bb')$.
	\end{itemize}

	On définit une relation d'équivalence $~$:
	$$ (a,b) \backsim  (a',b') \iff ab' = a'b $$

	Alors $K/\backsim $ (les classes d'équivalence de $E$ sur $\backsim$) est un corps et $A$ se plonge dans $K$ avec:
	$$\phi : a \in A  \mapsto \overline{(a,1)}$$
\end{definition}

\begin{remarque}
	$A, K  = Frac(A)$. Le plongement $\phi: A \to K$ nous permet d'identifier $A$ avec $\phi(A)$ de sorte que $A \subset K$.
	Ainsi un polynôme $P \in A[X]$  peut être vu comme un polynôme dans $K[X]$.
\end{remarque}

\begin{exemple}
	\begin{itemize}
		\item $Frac(\mathbb(Z)) = \mathbb{Q}$
		\item $Frac(\mathbb(K[X])) = K(X)$
	\end{itemize}
\end{exemple}

\begin{exemple}
	$2X^2 + 2X +2$ n'est pas irréductible dans $\mathbb{Z}[X]$ mais il est irréductible dans $\mathbb{Q}[X]$ car $2 \in \mathbb{Q}^\times$.
\end{exemple}



\begin{theorem}[Clasification des irréductibles] \href{https://fr.wikipedia.org/wiki/Lemme_de_Gauss_(polyn%C3%B4mes)#Applications}{Lemme de Gauss}\\
	Soit $A$ un anneau factoriel de corps de fraction $K$.
	Alors les irréductibles de $A[X]$ sont de deux types:
	\begin{itemize}
		\item Les polynômes constants $P = p$,  $p$ irréductibles dans $A$.
		\item Les polynômes primitifs de $\deg \geq 1 $ qui sont irréductibles dans $K[X]$.
	\end{itemize}
\end{theorem}


\begin{proof} On va traiter en premier les polynômes constants et après le reste.
	\begin{itemize}
		\item
		      Comme $A[X]^\times = A^\times$
		      si $P$ est constant, i.e. $P = p \in A$,
		      alors

		      $$ P \text{ inversible } \iff p \text{ inversible dans } A $$
		\item Montrons les deux implications pour les polynômes non constants.
		      \begin{itemize}
			      \item
			            Si $P$ primitif avec $\deg \geq 1$ dans $A[X]$ et irréductible dans $K[X]$, on écrit
			            $P = QR$, avec $Q,R \in A[X]$.

			            On a que $c(P) = c(Q) c(R) \in A^\times$ donc $c(Q) \in A^\times$ et $c(R) \in A^\times$. \\
			            La relation $P = QR \in K[X]$ implique que $Q$ ou $R$ sont de degré 0 (car $P$ primitif). \\
			            Comme ils sont primitifs, $Q \text{ ou } R \in A^\times \implies Q \text{ ou } R  \in A[X]^\times$
			            donc $P$ est irréductible dans $A[X]$.
			      \item
			            Soit $P$ irréductible de $A[X]$ avec $\deg \geq 1$. \\
			            Alors $c(P)$ divise $P$ donc $c(P) \in A^\times \implies P$ primitif.\\
			            Montrons maintenant que $P$ est irréductible dans $K[X]$.\\
			            On écrit $P = QR$ avec $Q,R \in K[X]$. On choisit $a,b \in A$ tel que $aQ \in A[X]$ et  $bR \in A[X]$.
			            On a donc que
			            \begin{eqnarray*}
				            abP &=& aQbR \\
				            c(aQ)c(bR) &=& c(abP) \\
				            &=& ab
			            \end{eqnarray*}
			            $$P = \frac{aQ}{c(aQ)} \frac{bR}{c(bR)} $$
			            donc $P$ produit de deux éléments de $A[X]$, donc, comme $P$ irréductible dans
			            $A[X]$, on a : $$ \deg (Q)= 0 \ \text{ou}\  \deg(R) = 0 $$
		      \end{itemize}
	\end{itemize}
\end{proof}




\section{Corps}


\begin{definition}
	Un corps est un anneau commutatif dans lequel tout élément non nul de $K^* = k \setminus \{0\}$ est inversible.
\end{definition}

\begin{definition}
	La caractéristique de $K$ est l'entier $n$ tel que le noyau du morphisme d'anneaux
	\begin{eqnarray*}
		\Z &\to& K \\
		1 &\mapsto& 1_K
	\end{eqnarray*}
	soit $\Zn{n}$, notée $car \ K$.
\end{definition}

\begin{exemple}
	$\Q, \R, \C$ sont des corps de caractéristique 0.
	\begin{itemize}
		\item Lorsque $p$ est premier, $\Zn{p}$ et $\Zn{p}(T)$(corps des fractions) est un corps pour tout $p$. Notons que $\Zn{p}(T)$ est infini.
		\item $\Q[i] = \left\{ a + bi, \text{ où } a,b \in \Q\right\} = Frac (\Z[i])$ est corps de $car\ 0$
	\end{itemize}
\end{exemple}

\subsection{Corps et espaces vectoriels}


\begin{definition}
	Soit $K$ un corps. Une extension de $K$ est un corps $L$ tel que $K$ soit un sous corps de $L$.
\end{definition}


\begin{remarque}
	Si $L$ est une extension de $K$, (noté $L/K$), alors $L$ est muni ipso facto d'une structure de $K$-espace vectoriel via la loi $\times$ dans $L$.\\
	En effet, la loi externe définissant la multiplication par un scalaire (élément de $K$) est la loi interne de multiplication dans $L$.

	D'autre part, si $\phi : K \to L$ est un morphisme de corps, il est injectif: \\
	Le noyau de $\phi$ est un idéal de $K$. Comme $K$ est un corps, ses seuls idéaux sont $\{0\}$ et $K$.\\
	On note qu'il ne peut pas être $K$ car $\phi(1_K) = 1_L$.

	Alors on peut identifier $K$ à $\phi(K)$. $L$ est une extension de $\phi(K)$. On étend la définition précédente en disant que $L$ est une extension de $K$.
\end{remarque}

\begin{example}
	$K(T)$ est une extension de $K$.
\end{example}

\begin{definition}
	Soit $L$ une extension de dimension finie sur $K$.\\
	Alors la dimension de ce $K$-espace vectoriel est un entier supérieur à 0 qu'on appelle
	le degré de $L$ sur $K$ que l'on note $[L : K]$. On dit dans ce cas que $L$ est fini sur $K$.
\end{definition}

\begin{theorem}[de la base télescopique]
	Soit $M$ un corps, $L$ un sous corps de $M$ et $K$ un sous corps de $L$.\\
	Alors lorsque $(e_i)_{i\in I}$ est une base de $L$ sous $K$ et $(f_i)_{i\in I}$ est une base de $M$ sous $L$, alors la famille
	$(e_if_j)_{(i,j)\in I\times J}$ est une base de $M$ sous $K$.
\end{theorem}


\begin{example}
	$\Q(\sqrt{2}) = \left\{ a + b\sqrt{2} \mid a,b \in \Q \right\}$ est une extension sur $\Q$ de degré 2.\\
	$M = \left\{ x + iy \mid x,y \in \Q(\sqrt{2}) \right\}$ est une extension $\Q(\sqrt{2})$ de degré 2.\\
	$M$ est une extension de $\Q$ de dimension 4 et une base est $\left\{ 1, i , \sqrt{2}, i\sqrt{2}\right\}$.
\end{example}


\begin{proof}
	\begin{itemize}
		\item Montrons que $(e_if_j)$ est libre. \\
		      Soient $\lambda_{i,j} \in K$ tels que
		      $$ \sum\limits_{i,j \in I \times J} \lambda_{i,j} e_i f_j = 0$$
		      et $(\lambda_{i,j})$ est une famille presque nulle. Alors  on a
		      $$ \sum\limits_{j \in J} \left( \sum\limits_{i \in I} \lambda_{i,j} e_i \right) f_j $$
		      $(f_j)$ est une famille libre de $M$ sous $L$, ce qui implique que pour tout $j \in J$.
		      $$ \sum\limits_{i\in I } \lambda_{i,j} e_i = 0$$
		      La liberté de la famille $(e_i)$ sous $K$ implique que pour tout $j \in J$ et tout $i \in I$, $\lambda_{i,j} = 0$.
		\item Montrons que $(e_if_j)$  est génératrice de $M$ sur $K$.\\
		      La famille $f_j$ est génératrice de $M$ sur $L$.\\
		      Soit $x \in M$, $\exists \, x_j$ une famille presque nulle de $L$ telle que $x = \sum\limits_{j\in J} x_j f_j$.\\
		      La famille $e_i$ est génératrice de $L$ sur $K$.\\
		      Pour tout $x_j \neq 0 $, $\exists (\lambda_{i,j})_{i \in I}$ presque nulle telle que $x_j = \sum\limits_{i \in I} \lambda_{i,j}e_i$.\\
		      Pour tout $j$ tel que $x_j = 0$, on choisit $\lambda_{i,j} = 0$. \\
		      On a $x = \sum\limits_{(i,j)\in I\times J} \lambda_{i,j} e_i f_j$, avec $(\lambda_{i,j})$ une famille presque nulle.
	\end{itemize}
\end{proof}


\begin{coro}[important]
	Si $L / K$ est fini et $M/L$ est fini, alors $M/K$ est fini et
	$$ [M:K] = [M:L] [L:K] $$
\end{coro}


\begin{definition}
	On considère $L/K$ une extension de corps et $\alpha \in L$. \\
	\begin{itemize}
		\item On note $K[\alpha]$ le sous-anneau de $L$ engendré par $K$ et $\alpha$. C'est aussi l'ensemble des $P(\alpha)$ avec $P \in K[X]$.
		\item $K(\alpha)$ le sous corps de $L$ engendré par $K$ et $\alpha$. C'est aussi l'ensemble des $F(\alpha)$ où $F \in K(X)$.
	\end{itemize}
\end{definition}


\begin{definition}
	Soit $L/K$ une extension de corps et $\alpha \in L$.\\
	On a un morphisme d'anneaux et de $K$-espaces vectoriels:
	\begin{eqnarray*}
		\phi: K[X] &\to& L \\
		P &\mapsto& P(\alpha)
	\end{eqnarray*}
	\begin{itemize}
		\item Si $\phi$ est injectif, alors $K[\alpha] \cong K[X]$ et $K(\alpha) \cong K(X)$. On dit que $\alpha$ est transcendant sur $K$.
		\item Si $\phi$ est non injectif, on note $\pi$ le générateur de $\ker\ \phi$. On dit que $\alpha$ est algébrique sur $K$ et on note $\pi$ le polynôme minimal de
		      $\alpha$ sur $K$. Ce polynôme est unique et unitaire.
	\end{itemize}
\end{definition}

\begin{remarque}
	Bien noter que les notions d'éléments algébriques ou transcendants dépendent du corps de base $K$. Tout élément de $L$ est algébrique sur $L$.\\
	Notons que $\pi$ est irréductible sur $K$. Comme $L$ est intègre, alors si le produit de deux polynômes de $K[X]$ s'annule en $\alpha$, alors l'un d'eux s'annule en $\alpha$.
\end{remarque}

\begin{example}
	\begin{itemize}
		\item $i$ est algébrique sur $\Q$ de polynôme minimal $X^2+1$, $L = \C, K = \Q$.
		\item Si $P \in K[X]$ unitaire tel que $P(\alpha) = 0$, alors $P$ est le polynôme minimal de $\alpha$ \ssi $P$ est irréductible sur $K$.
	\end{itemize}
\end{example}

\begin{remarque}
	Le nombre d'éléments de $\C$ algébriques sur $\Q$ est dénombrable. Donc il y a une infinité d'éléments transcendants sur $\Q$.\\
	$e$ est transcendant sur $\Q$.
\end{remarque}


\begin{prop}
	Soit $L/K$ une extension de corps et $\alpha \in L$.\\
	Il y a équivalence entre les assertions suivantes:
	\begin{itemize}
		\item  $\alpha$ est algébrique sur $K$
		\item $K[\alpha] = K(\alpha)$
		\item $K[\alpha]$ est un $K$-espace vectoriel de dimension finie.
	\end{itemize}
	Si on a l'une de ces assertions, alors l'entier $[K(\alpha) : K]$ est le degré du polynôme minimal de $\alpha$ sur $K$ et on l'appelle le degré $\alpha$ sur $K$.
\end{prop}

\begin{proof}
	\begin{itemize}
		\item $1 \implies 2$\\
		      Si $\alpha$ algébrique sur $K[\alpha] \cong K[X]/(\pi)$ et $\pi$ est un irréductible de $K[X]$ on a que $K[X]/(\pi)$ est un corps.\\
		      $K[X]$ est un corps qui est égal à son corps des fractions.
		\item Réciproquement, $\alpha$ est transcendant sur $K$.
		      $K[\alpha] \cong K[T]$ qui n'est pas un corps, donc on n'a pas $K[\alpha] = K(\alpha)$.
		\item $1 \implies 3$\\
		      Si $\pi$ est un polynôme minimal de $\alpha$ sur $K$, alors $K[\alpha] \cong K[X]/(\pi)$ qui est un $K$-espace vectoriel de dimension le degré de $\pi$.
		\item Réciproquement, \\
		      Si $\alpha$ est transcendant sur $K$, le $K$-espace vectoriel $K[\alpha]$ est isomorphe a $K[X]$ qui est de dimension infinie.
	\end{itemize}
	%TODO: Add explanation
\end{proof}

\begin{definition}
	Une extension $L/K$ est dite algébrique si tout élément de $L$ est algébrique sur $K$.\\
\end{definition}

\begin{remarque}
	Ainsi, toute extension finie est algébrique.\\
	Soit $L/K$ finie et $\alpha \in L$, $K[\alpha] \subset L$ et donc c'est un espace vectoriel de dimension finie. Donc d'après la proposition précédente, $\alpha$ est algébrique sur $K$.
\end{remarque}

\begin{theorem}
	Soit $L/K$ une extension de corps.\\
	On note $M$ l'ensemble des éléments de $L$ qui sont algébriques sur $K$.
	\begin{itemize}
		\item $M$ est un sous corps de $L$.
		\item Tout élément qui est algébrique sur $M$ est dans $M$. On dit que $M$ est la clôture algébrique de $L$ dans $K$.
		\item Si $L$ est algébriquement clos, $M$ est algébriquement clos. On dit que $M$ est la clôture algébrique de $K$.
	\end{itemize}
\end{theorem}



\begin{example}
	$\C$ est une clôture algébrique de $\R$.
\end{example}

\begin{example}
	$\C$ n'est pas une clôture algébrique de $\Q$. \\
	Si on note $\bar{\Q}$ l'ensemble des éléments de $\C$ algébriques sur $\Q$, on a que $\bar{Q} \neq \C$. On peut montrer que $\bar{\Q}$ est dénombrable.
\end{example}

\begin{proof}
	\begin{itemize}
		\item $0,1 \in M \implies K \subset M$ \\
		      \begin{itemize}
			      \item
			            Montrons que si $x \in M, x \neq 0, \  x^{-1} \in M \text{ et } -x \in M$. \\
			            Si $x \in M, \ x \in L$ et $P \in K[X], \ P \neq 0$ tel que $P (x) = 0$.\\
			            $x^{-1} \in L \text{ et } -x \in L$. Si $d = \deg P$ on a $Q(X) = X^d P(1/x) \in K[X]$ et $Q(x^{-1}) = 0$.\\
			            $S(x) = P(-x) \in K[x]$ et $S(-x) = 0$. \\
			            Ici on a construit explicitement les polynômes $Q$ et $S$ à partir de $P$.\\
			            Alors $x^{-1}$ et $-x \ \in M$
			      \item Montrons que si $x,y \in M, \  x + y \text{ et } x*y \in M$. \\
			            $K[x]$ est un $K$-espace vectoriel de dimension finie, \\
			            $y$ est algébrique sur $K$, donc a fortiori $K(x) =  K[x]$, donc $K[x][y]$ est de dimension finie.\\
			            $x+y $et $x*y\in K[x][y]$ donc $K[x+y]$ et $K[x*y]$ sont des $K$-espaces vectoriels de dimension finie car
			            ce sont des sous espaces vectoriels de $K[x][y]$
		      \end{itemize}
		      Alors $M$ est un sous-corps de $L$
		\item Soit $\alpha \in L$ algébrique sur $M$.\\
		      Il existe $P \in M[X]$
		      $$ P = X^na_n + \cdots + a_0$$
		      tel que $P(\alpha) = 0$. Comme chaque $a_j$ est algébrique sur $K$, on a par itération que $K'=K[a_0,\cdots, a_{n-1}]$ est un corps qui est une extension finie de $K$.\\
		      $K'[\alpha]$ est une extension finie de $K'$ puisque $P(\alpha)=0$ et$P \in K'[X]$. f
		      $$ \left[K[\alpha] : K\right]  =  \left[K[\alpha] : K'\right] \left[K' : K\right] < \infty $$
		      donc $\alpha$ est algébrique sur $K$ et donc $\alpha \in M$.
		\item
		      Si $P \in M[X]$ non constant, il admet une racine $\alpha \in L$ (car $L$ est algébriquement clos). \\
		      $\alpha$ est algébrique sur $M \implies \alpha \text{ algébrique sur K} \implies \alpha \in M$\ \\ % TODO: say that the First implies comes from the second point of the theorem.
		      $M$ est dont algébriquement clos.
	\end{itemize}
\end{proof}

\begin{remarque}
	Si $\alpha \in M$, $K[X]$ est un $K$-espace vectoriel mais la clôture algébrique sur
	$K$ n'est pas forcement un $K$-espace vectoriel de dimension finie.
\end{remarque}

\begin{remarque}
	$\bar{\Q}$ est dénombrable car $\Q$ est dénombrable.\\
	$\C$ n'est pas dénombrable ce qui démontre l'existence de nombres transcendants sur $\Q$.
\end{remarque}





\subsection{Corps de rupture et corps de décomposition}

Étant donnés $K$ un corps et $P\in K[X]$, on cherche une extension $L$ de $K$ telle que $P$ ait une racine dans $L$ ou toutes ses racines dans $L$.



\begin{definition}
	Soit $P$ un polynôme irréductible dans $K[X]$.\\
	On dit qu'une extension $L$ de $K$ est un corps de rupture de $P$ sur $K$ s'il existe une
	racine $\alpha \in L$ de $P$ telle que $L = K[\alpha] = K(\alpha)$. \\
	Ainsi, un corps de rupture est une extension dans laquelle $P$ a une racine et
	qui est minimale pour cette propriété.
\end{definition}

\begin{theorem}
	Pour tout polynôme irréductible $P$ de $K[X]$ il existe un corps de rupture $L$.
	De plus, $L$ est unique à un $K$-isomorphisme près.
\end{theorem}


\begin{remarque}
	L'unicité n'est pas vraie si $P$ est non irréductible.
	$$ P(X) = (X^2 - 2) (X^2 - 3) \in \Q[X]$$
	$\Q(\sqrt{2})$ et $\Q(\sqrt{3})$ ne sont pas $\Q$ isomorphes car
	$\Q(\sqrt{3})$ ne contient pas les racines de $X^2-2$.\\
	En effet, $\Q(\sqrt{3}) = \left\{  a + b\sqrt{3} \mid a,b \in \Q  \right\}$
	$$ (a + b\sqrt{2}) ^2 = a^2 + 2ab\sqrt{3}  + 3b \neq 2 \text{ sauf si } ab = 0$$
	Si $b=0, \ a = \sqrt{2} = \frac{p}{q}$ ce qui n'est pas possible. \\
	Si $a=0, \ b\sqrt{3} = \sqrt{2} \implies \sqrt{6} \in \Q $ ce qui n'est pas possible non plus.
\end{remarque}


\begin{proof}

	\begin{itemize}
		\item Comme $P$ est irréductible sur $K$ on a :
		      $$ L = K[X]/<P> \text{ est un corps car } K[X] \text{est principal}$$
		      $L$ est une extension de $K$ donc le plongement:
		      \begin{eqnarray*}
			      \phi: K &\to& L\\
			      \lambda   &\mapsto &\bar{\lambda}
		      \end{eqnarray*}
		      est un morphisme de corps.\\
		      Si on note $\alpha = \bar{X}$ la classe de $X$ dans $L$. $P(\alpha) = 0, L = K[\alpha]$
		      donc $L$ est un corps de rupture de $P$ sur $K$.
		\item Si $L$ est un corps de rupture de $P$ dans $K$ \\
		      Soit $\alpha'$ tel que $L' = K[\alpha']$ et $P(\alpha') = 0$, alors l'application
		      \begin{eqnarray*}
			      \phi : K[X] &\to& L'\\
			      Q &   \mapsto & Q(\alpha')
		      \end{eqnarray*}
		      et surjective de noyau $<P> (\ker \phi \supset <P>)$\\
		      Par un théorème de factorisation, on obtient un isomorphisme entre $L$ et $L'$ qui est un $K$-isomorphisme.
	\end{itemize}
\end{proof}

\begin{example}
	$\C$ est le corps de rupture de $X^2+1$ dans $\R$.\\
	$\Q(i)$ est le corps de rupture de $X^2+1$ dans $\Q$.\\
	$\Q(\sqrt[3]{2})$ est le corps de rupture de $X^3-2$ dans $\Q$.\\
	%TODO: Add explanation
\end{example}

\begin{remarque}[Important]
	Si $L$ est un corps de rupture pour $P$ dans $K$, alors:
	$$ [L : K] = \deg P$$
	%TODO: Add proof
\end{remarque}


\begin{definition}
	Soient $K$ un corps et $P\in K[X]$ (on ne suppose pas forcément $P$ irréductible).\\
	On dit qu'une extension $L$ de $K$ est un corps de décomposition pour $P$ sur $K$ si et seulement si $L$ vérifie les propriétés suivantes:
	\begin{itemize}
		\item $P$ est scindé sur $L$ (produit de polynômes de degré 1).
		\item $L$ est engendré comme corps (ou comme anneau) par les racines de $P$ sur $K$.
	\end{itemize}

	Ainsi un corps est décomposition est une extension minimale de $K$ pour laquelle $P$ est scindé.
\end{definition}

\begin{theorem}
	Pour tout $P\in K[X]$, il existe un corps de décomposition pour $P$ sur $K$ qui est unique à $K$-isomorphisme près.
\end{theorem}

\begin{proof}
	\begin{itemize}
		\item Existence:\\
		      On procède par récurrence sur le degré de $P$.
		      \begin{itemize}
			      \item Pour $\deg P \leq 1$, c'est évident: $L = K$
			      \item Soit $Q$ un facteur irréductible sur $K$ de $P$.
			            Comme $Q$ est irréductible, il admet un corps de rupture sur $K, \ K' = [x], \ Q(x) = 0$.\\
			            On écrit $P(X) = (X-x)P_1(X)$, avec $P_1 \in \K'[X]$. On a $\deg P_1 = \deg P -1$.\\
			            On peut alors appliquer l'hypothèse de récurrence à $P_1$ sur $K'$.\\
			            Il existe $L = K(x_2, \cdots,  x_n)$ tel que $P_1$ est scindé sur $L$ et
			            $x_2, \cdots, x_n$ sont des racines de $P$. On a donc $L = K(x, x_2, \cdots, x_n)$ est bien un
			            corps de décomposition pour $P$ sur $K$:
		      \end{itemize}
		\item Unicité: %TODO: Add proof

	\end{itemize}
\end{proof}




\section{Corps finis}

Un corps fini est un corps qui a un nombre fini d'éléments.

Sa caractéristique est forcément $p$ premier.
Si c'était 0, alors $\Z$ s'injecterait dans $K$ et $\Z$ serait infini.

$K$ peut être vu comme une extension de $\F_p$ via le morphisme:
\begin{eqnarray*}
	\F_p & \to & K \\
	\bar{1} &\mapsto& 1_K
\end{eqnarray*}

$K$ est en particulier un espace vectoriel de dimension finie ${K:\F_p}$ et $|K| = p ^{[K:\F_P]} $


Ici $\F_p = \left( \Zn{p}, +, \times \right)$, $p$ premier.
A isomorphisme près, il y a un seul corps à $p$ éléments.

$\F_p = \left( \Zn, +, \times \right)$ est un corps \ssi $n$ est premier.


\begin{theorem}
	Soit $q = p^n$, avec $p$ premier et $n \geq 1$. Alors il existe un corps de cardinal $q$ unique à isomorphisme près.
	C'est le corps de décomposition pour $X^q-X$ sur $\F_p$. On le note $\F_q$.
\end{theorem}


\begin{rappel} [Utile pour la preuve]
	Si $\alpha$ annule $P$ et sa multiplicité est supérieure ou égale à 2, alors $P'(\alpha) = 0$:
	$$P(X) = (X-\alpha)^2Q(X) \implies P'(X) = 2(X-\alpha)Q(X) + (X-\alpha)^2Q'(X) = 0$$
\end{rappel}


\begin{proof}
	\begin{itemize}
		\item Existence:\\
		      Soit $K$ le corps de décomposition de $X^q-C$ sur $F_p$. \\
		      On note $K'$ l'ensemble des racines dans $K$ de $X^q-X$. \\
		      $K'$ est en fait un corps: $0, 1 \in K'$. Si $x, y \in K'$'.\\
		      Montrons que $(x+y)^q = x^q + y^q = x+ y$.\\
		      Pour montrer l'identité précédente, on observe que pour tout $>0$ et pour tout $x,y \in K$ on a
		      $(x+y)^{p^k}= x^{p^k} + y^{p^k}$. Cela se montre par récurrence grâce à la formule $(x+y)^p= x^p + y^p$.\\
		      $p \mid \binom{p}{k}$ lorsque $1 \leq k \leq p-1$ aussi $(x+y)^p = \sum^p_{k=0} \binom{p}{k} x^k y^{p-k} = x^p + y^p$\\

		      $x\in K'$, alors $-x$ est aussi dans $K'$.
		      En effet, si $car K = 2$, $-x = x -2x = x$. \\
		      Si $car K$ est impaire, $(-x)^q = -x^q = -x$. \\
		      Évidemment $x,y \in K' \implies xy \in K'$, car $(x*y)^q = x^qy^q = xy$\\
		      $x\in K' x \neq 0 \implies \frac{1}{x} \in K'$.

		      Alors $K'$ est un sous corps de $K$.\\
		      Donc par la définition de corps de décomposition $K' = K$.\\

		      La dérivée de $X^q-X$ est $qX^{q-1} - X$ et $qX^{q-1} - 1 = 1$\\
		      Donc toutes les racines du polynôme sont simples et il y en a exactement $q$. $K$ est bien un corps de cardinal $q$.

		\item Unicité:\\
		      On considère $L$ un corps de cardinal $q$.\\
		      On sait que $\forall x \in L \setminus \{0\},\  x^{q-1} = 1$ (théorème de Lagrange appliqué au groupe multiplicatif $L\setminus \set 0$ et
		      cardinal $q-1$), donc:
		      $$\forall x \in L, \ x^q -x = 0$$
		      $X^q-X$ est scindé dans $L$ (car il a $q$ racines distinctes).\\
		      $L$ contient un corps de décomposition pour $X^q-X$ sur $\F_p$. Autrement dit, un $K_1$ qui est
		      isomorphe à $K$ (d'après les propriétés de décomposition).\\
		      $L_1 \subset L$ et $|K_1| = q = |L| \implies K_1 = L \cong K$.
	\end{itemize}
\end{proof}

\begin{example}

	\begin{itemize}
		\item $P = X^2+X+1$ est irréductible sur $\F_2$ et
		      $$\F_2[X]/ <x^2+x+1> \cong F_4$$

		      On prend $\alpha$ une racine de $x^2+x+1$.
		      $$\F_2[X]/ <x^2+x+1> = \set{ a + \alpha b \mid a,b \in \F_2} \cong \F_4$$
		      La table de multiplication:

		      \begin{center}
			      \begin{tabular}{c|c|c|c|c}
				      $*$        & 0          & 1        & $\alpha$    & $1+\alpha$  \\
				      \hline
				      0          & 0          & 1        & $\alpha$    & $1+\alpha$  \\
				      \hline
				      1          & 1          & 1        & $\alpha$    & $\alpha +1$ \\
				      \hline
				      $\alpha$   & $\alpha$   & $\alpha$ & $\alpha +1$ & 1           \\
				      \hline
				      $1+\alpha$ & $1+\alpha$ & $\alpha$ & 1           & $\alpha$    \\
			      \end{tabular}
		      \end{center}

		      Car on a $\car \ \F_4 = 2$ et donc $\alpha^2 = (\alpha^2 + \alpha + 1 ) + \alpha + 1 =  P(\alpha) + \alpha + 1 = \alpha + 1$. \\
		      De plus $\alpha(\alpha + 1) = \alpha^2 + \alpha = \alpha + 1 + \alpha = 1$.
		\item$x^3+x+1$ est irréductible sur $\F_2$ et $\F_2[X]/ <x^3+x+1> \cong F_6$.
		\item$ \F_3[X]/ <x^2+1> \cong F_9$
	\end{itemize}
\end{example}


\begin{exercice}
	Si $K$ est un corps fini et $P \in K[X]$ irréductible sur $K$, alors le
	corps de rupture de $P$ sur $K$ est aussi un corps de décomposition pour $P$ sur $K$.
\end{exercice}

\begin{remarque}
	$\F_{p^n}$ est une extension de $\F_{p^m}$ \ssi $m$ divise $n$.\\
	Ainsi $\F_8$ n'est pas une extension de $\F_4$. \\
\end{remarque}

\begin{proof}
	\begin{itemize}
		\item $\degExt {F_{p^n}}{F_{p^m}} = d \in \N$\\
		      $p^n = |F_{p^n}| = |F_{p^m}|^d = p^{md} \implies n = md$.

		\item Réciproquement, si $m \mid n$, on écrit $n = md$\\
		      On a $X^{p^m - 1} - 1$ divise $X^{p^n-1} - 1$  et $X^{p^m} - X$ divise $X^{p^n} - X$\\
		      Toutes les racines de $X^{p^m} - X$ sont racines de $X^{p^n} - X$, donc $F_{p^m} \subset F_{p^n}$.
	\end{itemize}
\end{proof}



\begin{remarque}[Morphisme de Frobenius]
	Soit $K$ un corps de caractéristique $p >0$ et $\psi : K \to K$ définie par $\psi(x) = x^p$.\\
	Alors $\psi$ est un morphisme de corps, donc injectif. Si $K$ est fini, alors $\psi$ est un automorphisme de corps.\\
	Ainsi $\psi(x + y) = (x+y)^p = x^p + y^p, \forall x,y \in K$.\\
	De plus $\set {x \in K \mid x^p = x} = \F_p$.\\
	Cela fournit une famille de morphismes $K \to K$:
	$$ \psi,\ \underbrace{\psi \circ \cdots \circ \psi}_{n \text{ fois}} = \psi^n : \begin{cases}
			K \to K \\
			x \mapsto x^{p^n}
		\end{cases}$$

	Pour $P \in F_q[X], \ q = p^n$, on a $\psi(P(x)) = P (\psi(x))$, car $F_q$ est de caractéristique $p$.
\end{remarque}


\subsection{Polynômes irreductibles sur un corps fini}


Soit $I(n,q)$ le plus petit polynôme unitaire de degré $n$ irréductible sur $\F_q$ dans $\F_q[X]$.

\begin{theorem}
	Pour tout $q$ puissance d'un nombre premier et $n \geq 1$ on a $I(n,q) > 0$.\\

	Plus précisément:
	$$ I(n,q) = \frac{1}{n} \sum_{d|n} \mu(\frac{n}{d}) q^d$$

	où $\mu$ est la fonction de Möbius:

	\[ \mu(k) = \left\{\begin{array}{ll}
			0      & \text{si il existe } l \text{ premier tel que } l^2 \text{ divise } k                            \\
			(-1)^r & \text{si } k = p_1 \cdot p_2 \cdot \ldots \cdot p_r \text{ avec } p_i \text{ premiers distincts}
		\end{array} \right.\]

\end{theorem}

\begin{lemma}
	$$\sum_{d|n} \mu(d) = \left\{\begin{array}{ll}
			1 & \text{si } n = 1 \\
			0 & \text{sinon}
		\end{array} \right.$$
\end{lemma}

\begin{proof}
	On peut se ramener à $n$ sous facteur carré $n = p_1 \ldots p_r$ avec $p_i$ premiers distincts.\\
	\begin{eqnarray*}
		\sum_{d|n} \mu(d) &=& \sum_{j=0}^r (-1)^j \#\set{d \mid n : \text{nombre de facteurs premiers de } d \text { est } j} \\
		&=& \sum_{j=0}^r (-1)^j \binom{r}{j} \\
		&=& (1-1)^r = 0 \text{ si } r > 1
	\end{eqnarray*}
\end{proof}

\begin{lemma}
	Soient $f$ et $g$ des applications de $\N^*$ dans $\C$, alors:
	$$ \forall n \in \N^*, g(n) = \sum_{d|n} f(d) \iff \forall n \in \N^*, f(n) = \sum_{d|n} \mu(\frac n d) g(d) $$

\end{lemma}

\begin{proof}
	Si $g(n) = \sum_{d|n} f(d)$ alors:
	\begin{eqnarray*}
		\sum_{d|n} \mu(\frac n d) g(d) &=& \sum_{d|n} \mu(\frac n d) \sum_{l|d} f(l) \\
		&=& \sum_{l|n} f(b) \underbrace{\sum_{d|n,\ l|d}}_{ d = ld', \ d' | \frac n l} \mu(\frac n d) \\
		&=& \sum_{l | n} f(l) \sum_{d' | \frac n l} \mu(\frac n {ld'}) \\
		&=& \sum_{l | n} f(l) \underbrace{\sum_{m | \frac n l} \mu(m)}_{ 0, \text{ sauf si } l = n } \\
		&=& f(n)
	\end{eqnarray*}
\end{proof}

\begin{proof}[du théorème]
	Il suffit de montrer que $q^n = \sum_{d|n} d I(d,q)$.\\

	Considérons $Q = X^{q^n} - X$ et sa décomposition en produits de polynômes irréductibles unitaires:
	$$ Q = P_1 \ldots P_r$$

	Tous les $p_j$ sont distincts car si $P_i = P_j (i \neq j)$.\\
	$$P_j^2 | Q \implies P_j | Q' = - 1$$
	Montrons que les $P_j$ sont exactement les polynômes irréductibles unitaires dont le degré divise $n$. Soit $P$ un
	polynôme irréductible de degré $d$. Supposons $d | n$. Soit $x$ dans un corps de rupture pour $P$ tel que $P(x) = 0$.
	Ce cors est de degré $d$ sur $\F_q$. Il est donc isomorphe à $\F_{q^d}$. Donc $x^{q^d} = x$.\\
	Avec $d|n$, on a $x^{q^n} = x$. Soit $m = kd$. Si $\psi_{q^d}$ est le morphisme de Frobenius $x \mapsto x^{q^d}$, alors
	$$\psi_{q^d}^k(x) = \psi_{q^d} \circ \dots \circ \psi_{q^d}(x) = x^{q^n} = x$$
	On a $Q(x) = 0$. Comme $P$ est le polynôme minimal de $x$ sur $\F_q$, on a donc $P | Q$.\\

	Réciproquement, si $P | Q$ de degré $d$.\\
	Toute racine $x$ de $P$ anule $Q$, donc $\F_{q^d}$ est le corps de décomposition de $\Q$ qui contient $\F_{q^d} = F_q[x]$, donc
	$d | n$ (comme pour le théorème de la base télescopique $\F_q \subset \F_{q^d} \subset \F_{q^n}$).\\

	En calculant le degré de $Q$ on en déduit que
	$$q^n = \sum_i \deg P_i = \sum_{d|n} d I(d,q)$$
\end{proof}


\begin{example}
	$ q = 2, \ n = 4, \ 2^4 = 16 $
	$$	X ^16 - X = X(X-1)(X^2 + X + 1)(X^4 + X + 1)(X^4 + X^3 + 1)(X^4 + x^3 + x^2 + x + 1) $$
	$$	I(2,2) = \frac{1}{2}\sum_{d|2} \mu(\frac{2}{d}) 2^d = \frac{1}{2}(4-2) = 1$$
	$$	I(4,2) = \frac{1}{4}\sum_{d|4} \mu(\frac{4}{d}) 2^d = \frac{1}{4}(2^4 - 2^2) = 3$$
\end{example}



\begin{coro}
	Pour tout $n$ et tout $q$ puissance d'un nombre premier:
	$$ I (n, q) > 0 $$
\end{coro}


\begin{proof}
	\begin{eqnarray*}
		n I(n, q) & = & \sum_{d|n} \mu(\frac{n}{d}) q^d \\
		&>& q^n - \sum_{d|n,\ d < n} q^d \\
		&>& q^n - \sum_{1 \leq d\leq \lfloor \frac n 2 \rfloor} q^d \\
		&=& q^n - q\frac{q^{\lfloor \frac n 2 \rfloor} - 1}{q-1} \\
		&>& 0
	\end{eqnarray*}
\end{proof}


\subsection{Critères de réductibilité sur $\Q$ et $\F_p = \Zn p$}
Pour étudier la réductibilité d'un polynôme sur $\Q$ on peut toujours se ramener à un
polynôme sur $\Z$ primitif.



\begin{prop}
	Soit $P = a_nX^n + \ldots + a_0 \in \Z[X]$.\\
	Soit $p$ un nombre premier tel que $p \nmid a_n$.\\
	Si $\bar{P}$,la réduction modulo $p$, est irréductible sur $\F_p$, alors $P$ est irréductible sur $\Q$.
	De plus, si $P$ est primitif, alors $P$ est irréductible sur $\Z$.
\end{prop}


\begin{remarque}
	$a \nmid b$ est essentiel : $2X^2 - X + 1$ est irréductible sur $\F_2$ mais réductible sur $\Q$, 1 est une racine.
\end{remarque}

\begin{remarque}
	Ce critère est une condition suffisante mais pas nécessaire.\\
\end{remarque}


\begin{proof}
	On suppose $P$ primitif, réductible sur $\Q$ et donc aussi sur $\Z$.\\
	Alors il existe $R, S \in \Q[X]$ , tels que $P = RS$
	$$ \exists a, b \in \N* \text{ tels que } aR, bS \in \Z[X]$$
	$$ abP = aRbS$$
	$$ c(abP) = ab = c(aR)c(vS) $$
	$$P =   \frac{c(aR)}{c(aR)} \frac{c(bS)}{c(bS)}$$
	Modulo $p$ $\bar{P} = \bar{Q}\bar{R}$.
	De plus, on a:
	\begin{itemize}
		\item $\deg Q = \deg \bar{Q}$
		\item $\deg R = \deg \bar{R}$

	\end{itemize}
	car leurs coefficients dominants ne divisent pas $p$.\\
	Donc $\bar{P} = \bar{Q}\bar{R} \implies \bar{P}$ n'est pas irréductible dans $\Zn{p}[X]$.\\
	On a donc montré la contraposée de la proposition.
\end{proof}


\begin{prop}
	Soit $P \in K[X]$ et $\deg P = n$, Si $P$ n'a pas de racines dans toute extension
	de $K$ de degré au plus $\frac{n}{2}$, alors $P$ est irréductible sur $K$.
\end{prop}

\begin{remarque}
	Si $n = 2$ ou $n = 3$ ce résultat dit que si $P$ n'a pas de racines dans $K$, alors $P$ est irréductible.
\end{remarque}


\begin{proof}
	Soit $P$ réductible sur $K$. Alors il existe $Q$ irréductible de degré $\leq \frac{n}{2}$ qui divise $P$.\\
	Soit un corps de rupture pour $Q$ sur $K$.\\
	$$\degExt L K = \deg Q \leq \frac{n}{2}$$
	$L$ contient une racine de $Q$ et donc une racine de $P$.
\end{proof}



\begin{example}
	$P(X) = X^4 + 1 \in \F_p[X]$ avec $p$ premier n'est pas irréductible sur $\F_p$ alors que $X^4 + 1$ est irréductible sur $\Q$.

	Montrons que $P$ a toujours une racine dans $\F_{p^2}$ :
	$F_{p^2}$ est le corps de décomposition de $X^{p^2} - X = X (X^{p^2-1} - 1)$ sur $\F_p$.
	Donc les éléments non nuls de $F_{p^2}$ sont les racines de $X^{p^2-1} - 1$.\\
	$p$ est impair, donc $p^2 - 1 \equiv 0 \mod 8$ car $1,3,5$ et $7$ au carré donnent $1$ modulo $8$.\\

	On choisit $x$ une racine de $X^{p^2-1} - 1$ d'ordre $8$ mais qui ne soit pas d'ordre $4$.

	$x^8-1 = 0 = (X^4-1)(X^4+1)$, donc $x^4 + 1 = 0$.

	Donc $P$ a une racine dans $\F_{p^2}$ et donc $P$ est réductible sur $\F_p$.
\end{example}





\section{Réductions d'endomorphisme et polynômes d'endomorphisme}

$E$ désignera un $K$-espace vectoriel.


\subsection{Polynômes d'endomorphisme}
On note $\LE$ l'ensemble des endomorphismes de $E$.

\begin{definition}
	Soit $u \in \LE$. Pour tout polynôme $P \in K[X] = a_0 + a_1X + \ldots + a_nX^n$, on définit $P(u)$ par:
	$$P(u) = a_0 Id_E + a_1 u + \ldots + a_n u^n$$
	L'application $P \mapsto P(u)$ est un morphisme d'algèbres (d'anneaux) de $K[X]$ dans $\LE$.
	$$(PQ)(u) = P(u)\circ Q(u)$$
\end{definition}

\begin{prop}
	Soit $u \in \LE$ et $P, Q \in K[X]$. Alors:
	$P(u)$ et $Q(u)$ commutent: $P(u)\circ Q(u) = Q(u)\circ P(u)$
\end{prop}

\begin{proof}
	Grâce à la linéarité, il suffit de le montrer pour $P = X^m$ et $Q = X^n$.
	$$u^m\circ u^n = u^{m+n} = u^{n+m} = u^n\circ u^m$$
\end{proof}

\subsection{Polynômes annulateurs et polynôme minimal}


\begin{definition}
	Soit $y \in \LE$ et$P \in K[X]$. On dit que $P$ est un polynôme annulateur de $u$ si $P(u) = 0_{\LE}$
\end{definition}


\begin{prop}
	Tout endomorphisme dans un espace vectoriel de dimension finie admet un polynôme annulateur non nul.
\end{prop}


\begin{proof}
	$\set{u^k \mid 0 \leq k \leq n^2}$ est une famille liée car $\dim \LE = n^2$.\\
\end{proof}


\begin{remarque}
	Le théorème de Cayley-Hamilton nous dira que le polynôme caractéristique est un polynôme annulateur.
\end{remarque}

\begin{remarque}
	Si $E$ n'est pas de dimension finie, cela n'est pas vrai en general:

	Soit $ D : K[X] \to K[X]$ l'application qui à un polynôme associe sa dérivée. Alors $D$ n'a pas de polynôme annulateur.\\
	Soit $Q = \sum\limits_{i=0}^n a_iX^i \in K[X]$.
	$$Q(D)(P) = \sum\limits_{k=0}^n ia_i P^{(k)}$$
\end{remarque}



\begin{prop}
	L'ensemble $I_u$ des polynômes annulateurs de $u \in \LE$ est un idéal de $K[X]$.
	Si $I_u \neq \set{0}$, alors il admet un générateur unique appelé polynôme minimal de $u$ et noté $\mu_u$.
\end{prop}


\begin{definition}
	Un endomorphisme $u$ est dit nilpotent s'il existe $n \in \N$ tel que $u^n = 0_{\LE}$.\\
	Dans ce cas, le polynôme minimal de $u$ est de la forme $X^n$ et on appelle $n$ l'indice de nilpotence de $u$.
\end{definition}


\begin{remarque}
	Si $u$ admet un polynôme minimal $\mu_u$ de degré $d$, alors $dim K[u] = d$.
\end{remarque}

\begin{proof}
	%TODO
\end{proof}

\begin{example}
	Si $u$ est une homothétie de rapport $\lambda$, autrement dit si $u = \lambda id_E$, alors $\mu_u = X - \lambda$.
\end{example}

\begin{remarque}
	On dit que $p$ est un projecteur \ssi ${p}^2 = p$.
	$\mu_p = X^2 - X$ sauf si $p = id_E$ ou $p = 0_{\LE}$.
\end{remarque}

\begin{prop}
	Soit $u \in \LE$ admettant un polynôme minimal $\mu_u$, alors $u$ est inversible \ssi $\mu_u(0) \neq 0$
\end{prop}

\begin{proof}
	%TODO
\end{proof}

\subsection{Lemme des noyaux}

\subsubsection{Etude du $\ker P(u)$}


\begin{prop}
	Soient $P,$ et $Q \in K[X]$ tels que $\pgcd{P}{Q} = D$ et $u \in \LE$.
	Alors
	$$\ker P(u) \cap \ker Q(u) = \ker D(u).$$
\end{prop}

\begin{proof}
	\begin{itemize}
		\item $D \mid P \implies \ker D(u) \subset \ker P(u)$\\
		      En effet, $P = RD$, $P(u)(x) = R(u)(D(u)(x)) = 0$.\\
		      Donc $D(u)(x) = 0 \implies P(u)(x) = 0$
		      $$ \ker D(u) \subset \ker P(u) \cap \ker Q(u).$$

		\item Montrons l'autre inclusion.\\
		      Le théorème de Bézout affirme l'existence de $U$ et $V \in K[X]$ tels que $U(X)P(X) + V(X)Q(X) = D(X)$.\\
		      Donc $\forall x \in E$,
		      $$ D(u)(x) = U(u)P(u)(x) + V(u)Q(u)(x) = 0$$
		      Donc $x \in \ker P(u) \cap \ker Q(u) \implies x \in \ker D(u)$.
	\end{itemize}
\end{proof}

\begin{coro}
	Soit $u \in \LE$ de polynôme minimal de $\mu_u$.
	Soit $P \in K[X]$.
	$$\ker P(u) = \ker D(u)$$
	Où $D = \pgcd{P}{\mu_u}$.
\end{coro}

\begin{remarque}
	Ce résultat nous permet de nous restreindre à des $P$ tels que $P \mid \mu_u$.
\end{remarque}

%TODO: add example showing that \mu_u in not necessarily irreducible

\begin{proof}
	Par définition $\ker \mu_u(u) = E$, car $\mu_u(u) = 0_{\LE}$, donc
	\begin{eqnarray*}
		\ker P(u)  &=& \ker P(u) \cap E \\
		&=& \ker P(u) \cap \ker \mu_u(u) \\
		&=& \ker D(u)
	\end{eqnarray*}
\end{proof}

\begin{coro}
	Soit $u \in \LE$ de polynôme minimal $\mu_u$ et $P \in K[X]$ unitaire, avec $P \mid \mu_u$.\\
	Soit $v = \restr u {\ker P(u)}$.\\
	Alors $\mu_v = P$.
\end{coro}


\begin{proof}
	Pour tout $P$, $\ker P(u) $ est stable par $u$.\\
	En effet, si $x \in \ker P(u)$, alors $P(u)(x) = 0$, donc
	\begin{eqnarray*}
		P(u)(u(x)) &=& (P (u) \circ u)(x) \\
		&=& (u \circ P(u))(x) \\
		&=& u(P(u)(x)) \\
		&=& 0_{\LE}
	\end{eqnarray*}
	Donc $u(x) \in \ker P(u)$.\\ %TODO: Maybe add remark about (XP(X))(u)...

	Montrons que $F = \ker P(u), \ \forall x \in F, \ P(v(x)) = P(u)(x) = 0$.\\

	$P(0)=0_{\LE}$, $\mu_v \mid P$.\\
	Prenons $Q \in K[X]$ tel que $\mu_v = QP$.\\
	$\mu_v(u) = 0_{\LE} \implies \ker Q(u) \subset \ker P(u)$.\\
	%TODO: Finish
\end{proof}



\subsubsection{Lemme des noyaux}

Soit $u \in \LE$.\\
Il permet de décomposer un espace vectoriel en somme directe d'espaces vectoriels stables par $u$ et adaptés à la réduction de $u$.

\begin{lemma}[des noyaux]
	Soit $(P_k)$ une famille de polynômes 2 à 2 premiers entre eux.
	Soit $u \in \LE$.
	$$ \ker \left(\left(\prod_{k=1}^N P_k \right)(u)\right) = \bigoplus_{k=1}^N \ker P_k(u)$$

	De plus, la projection de $\ker \left(\left(\prod\limits_{k=1}^N P_k \right)(u)\right)$ sur l'un des  $\ker P_j(u)$ parallèlement à la somme des autres est un polynôme en $u$.
\end{lemma}

\begin{proof}
	Par récurrence sur $N$.

	\begin{itemize}
		\item Montrons que si $P_1$ et $P_2$ sont premiers entre eux, alors $\ker (P_1P_2) (u) = \ker P_1(u) \oplus \ker P_2(u)$.\\
		      %TODO
	\end{itemize}
\end{proof}


\subsubsection{Conséquence: Décomposition en sous espaces vectoriels stables}


Soit $u \in \LE$ avec un polynôme minimal $\nu_u$. On décompose $\nu_u$ en produit
d'irréductibles:
$$ \nu_u = \prod\limits_{k=1}^N p_k^{\alpha_k} $$
où $\alpha_k \geq k$, les $p_k$ sont irréductibles unitaires et 2 à 2 distincts.


\begin{coro}
	$$E = \bigoplus_{k=1}^N \ker\left( {P_k^{\alpha_k}(u)} \right)$$
\end{coro}

\begin{proof}
	La preuve consiste à utiliser le lemme des noyaux, sachant que
	$$\ker \mu_u(u) = E $$.
    Les sous-espaces vectoriels  $\ker\left( {P_k^{\alpha_k}(u)} \right)$ sont stables par $u$.
	Il suffit pour réduire $u$ de se restreindre à chacun de ces sous espaces vectoriels.
\end{proof}


\subsection{Rappels d'algèbre linéaire}

\subsubsection{Déterminants} %Maybe don't use subsubsections


Soit $n = \dim E$, pour tout $u \in \LE$, on peut associer $\det u \in K$.

Il y a plusieurs façons de le définir.


\begin{itemize}
	\item De manière canonique : pour toute forme $n$ linéaire alternée $f$ sur $E$ on a:
	      $$ f \circ u = \det (u) * f$$
	      En effet, l'espace de ces formes $n$ linéaires alternées est de dimension 1. De cette manière, $f$ et $f \circ u$ sont proportionnelles.

	      On qualifie cette définition de canonique car elle ne nécessite pas le choix d'une base de $E$.

	\item Soit de manière matricielle : $\det u$ est un polynôme explicite en les coefficients de la matrice de $U$ prise dans une base de $E$.
	      Ce déterminant ne dépend pas du choix de la base.
\end{itemize}


D'après 1, $det (Id_E) = 1$.

Si $u$ et $v \in \LE$,
\begin{eqnarray*}
	f \circ u \circ v &=& \det (u \circ v) f \\
	&=& \det (v) f  \circ u \\
	&=& \det (v) \det (u) f
\end{eqnarray*}

Alors,
$$\det (u \circ v) = \det (u) \det (v) = \det (v \circ u) $$

En particulier, si $v$ est inversible, alors
$$ \det (v^{-1}) = (\det v )^{-1} $$

Le déterminant induit un morphisme de groupe :


\begin{eqnarray*}
	\det (GL(E), \circ) &\to& (K^{\times}, *)\\
	u &\mapsto& \det (u)
\end{eqnarray*}

$u$ est inversible \ssi $det(u) \neq 0$

\subsubsection{Polynôme caractéristique d'un endomorphisme}

On apelle polynôme caractéristique de $u$ le polynôme  $\chi_u$ défini par
$$\chi_u(X) = \det (X Id_e - u)$$

On fixe une base $B$ de $E$, on définit $M = Mat_B u$.
Dans ce cas $\chi_u(X) = \det (CI_n -M)$
(c'est un polynôme en $X$ de degré $n$ unitaire).

La définition de $\chi_u(X) = \det (XI_n-M)$ ne dépend pas du choix de la base.
Si $P$ inversible, alors:


\begin{eqnarray*}
	\det (XI_n - PMP^{-1}) &=& \det (P(XI_n -M) P^{-1})\\
	&=& \det (P) \det (X-I_nM)\det P^{-1}\\
	&=& \det (XI_n -M)
\end{eqnarray*}



De cette manière, si $v$ inversible:
$$ \chi_{v\circ u \circ v^{-1}} = \chi_{u}$$


\subsubsection{Notion de valeur propre}

$\lambda \in K$ tel que $\exists x \in E$ non nul vérifiant $u(x) = \lambda x$.

\subsubsection{Notion de vecteur propre associé à une valeur propre}

$$E_n = \set{ x \in E: u (x) = \lambda x } = \ker (u - \lambda Id_E) $$

\begin{prop}
	Les vecteurs propres associés à des valeurs propres 2 à 2 disjoints forment une famille libre.
\end{prop}

\begin{coro}
	Soit $E$ de dimension finie $n$ et $u \in \LE$, alors $u$ a au plus $n$ valeurs propres distinctes.
\end{coro}


\begin{prop}
	Les sous-espaces vectoriels correspondant à deux valeurs propres distinctes sont en somme directe.

	Autrement dit,
	$$E_{\lambda_1} \cap E_{\lambda_2} = \set{0} \text { si } \lambda_1 \neq \lambda_2$$
\end{prop}

\begin{prop}
	Soit $u \in \LE$ et $P \in K[X]$: Pour toute valeur propre $\lambda$ de u:
	$$ P(u) = 0_{\LE} \implies P(\lambda) = 0_K$$
\end{prop}

\begin{proof}
	Si $x \in E_\lambda , P(u)(x) = P(\lambda) x$ %TODO: add yellow explanation
\end{proof}


\begin{prop}
	Soit $u \in \LE$ de polynôme minimal $\nu_u$,
	$$ \lambda \text{ valeur propre de u }\iff \mu_u(\lambda) = 0$$
\end{prop}


\begin{proof}
	\begin{itemize}
		\item  $\Rightarrow$:
		      $\lambda$ valeur propre de $u$ (on dit aussi $\lambda \in \spec(u)$)
		      $$\exists x \in E \text { non nul tel que } u(x) = \lambda x \ \nu_u(u)(x) = \nu_u(\lambda)x =0$$
		\item $\Leftarrow$:
		      $\nu_u(\lambda) = 0 \implies \exists Q \in K[X]$ tel que $\nu_u(X) = (X - \lambda)Q(X)$\\
		      Si $\lambda$ n'était pas une valeur propre, alors $u-\lambda Id_E$ serait inversible
		      $$ (u-  \lambda Id) \circ Q = 0 \iff Q(u) = 0_{\LE}$$
		      Ce qui contredit la minimalité de $\nu$.
	\end{itemize}
\end{proof}



\begin{prop}
	Soit $u \in \LE$, $E$ de dimension $n$.
	$F$ un sous-espace vectoriel de $E$ stable par $u$.
	On peut définir $v \in \LE$ par $v = \restr u F$. On a alors:
	$$\chi_v \text{ divise } \chi_u$$
\end{prop}

\begin{proof}
	%TODO
\end{proof}

\begin{prop}
	Soit $u \in \LE$ et $E = F \oplus G$ avec $F$ et $G$ stables par $u$ avec $n = \dim E$.

	$$\chi_u (X)= \chi_v (X) \chi_w(X)$$
	avec $v = u_{|F}$ et $w = u_{|G}$
\end{prop}

\begin{proof}
	La même que précédemment, on posant $(f_{p+1}, f_n)$ une base de $E$.
	Dans ce cas-là, $B = 0$, $D = mat_{(f_{p+1}, \dots, f_n)} w$
	$det(XI_{n_p} - D) = \chi_w(X)$
\end{proof}


\subsection{Endomorphismes cycliques}

Soit $u \in \LE$ et $x \in E$ non nul.
Le sous espace vectoriel cyclique $E_{u,x}$ est le plus petit sous espace vectoriel de $E$ contenant $x$ et stable par $u$.

Il est engendré par les $u^k(x), \ k \in \N$ et si la dimension de cet espace est $p$, alors la famille
$ (x, u(x), \ldots, u^{p-1}(x))$ est une base de $E_{u,x}$.

\begin{proof}
	\begin{itemize}
		\item Soit $F \subset vect (x, u(x), \ldots, u^{p-1}(x))$.\\
		      $x \in F$ et $F$ est stable par $u$.\\
		      Donc $E_{u,x} \subset F$.

		\item Réciproquement, les $u^k(x) \in E_{u,x}$ car $x \in E_{u,x}$ et $E_{u,x}$ est stable par $u$.
		      Donc $F \subset E_{u,x}$.
	\end{itemize}
	On en conclut que $E_{u,x} = vect (x, u(x), \ldots, u^{p-1}(x))$.
\end{proof}

On désigne par $\mu_{u,x}$ le polynôme minimal de $u$ en $x$, c'est à dire le générateur unitaire de l'idéal
$$ \set{P \in K[X] \mid P(u)(x) = 0_E}$$

Comme $\dim E$ est fini cet idéal est \textbf{non} réduit à $\set{0}$.


Si $\mu_{u,x} = X^d + \sum_{k=0}^{d-1} a_k X^k$ alors la famille $(x, u(x), \ldots, u^{d-1}(x))$ est libre.

La famille est génératrice.\\
En effet, si on écrit $v = \sum \beta_k u^k(x) = P(u)(x)$, avec $P = \sum \beta_k X^k$.

On fait la division euclidienne de $P$ par $\mu_{u,x}$:

$$ P = Q \mu_{u,x} + R, \ \deg R \leq d -1$$
%TODO Improve this
$$ v = P(u)(x) = Q(u) \mu_{u,x}(u)(x) + R(u)(x) = R(u)(x)$$

Car $\mu_{u,x}(u)(x) = 0$ par définition.

$$ P(u)(x) \in vect (x, u(x), \ldots, u^{d-1}(x))$$

On a aussi $(u^k(x))_{k \in \N}$ base de $E_{u,x}$, donc en particulier, $ d = \dim E_{u,x} - p $.

\begin{definition}
	$u \in \LE$ est dit cyclique \ssi $\exists x \in E$ tel que $E = E_{u,x}$.
\end{definition}

\begin{example}
	Si $\dim E = 2$, $u$ est soit une homothétie, soit un endomorphisme cyclique.

	\begin{itemize}
		\item Soit pour tout $x \in E$ non nul, $\dim (x, u(x)) = 1$.

		      $\forall x \in E, \exists \lambda_x \in K, u(x) = \lambda_x x$.

		      \begin{itemize}
			      \item $\lambda_x$ ne dépend pas de $x$. En effet, si $x$ et $y$ sont non colinéaires:
			            $$ u (x+y) = \lambda_{x+y} (x+y) = \lambda_x x + \lambda_y y \implies (\lambda_{x+y} - \lambda_x) x = (\lambda_{x+y} - \lambda_y) y$$
			            Donc $\lambda_{x+y} = \lambda_x = \lambda_y$.

			      \item Si $x$ et $y$ sont colinéaires, alors $\exists z$ non colinéaire à $x$ et donc à $y$. D'après ce qui précède, $\lambda_x = \lambda_z = \lambda_y$.
		      \end{itemize}

		      Donc $\lambda_x = \lambda$ ne dépend pas de $x$, et donc $u$ est une homothétie.

		\item Sinon $\exists x \in E$ tel que $(x, u(x))$ est une base de $E$ et dans ce cas $u$ est cyclique
	\end{itemize}
\end{example}

\begin{example}
	Soit $u \in \LE$, $\dim E = n$, On suppose que $u$ admet $n$ valeurs propres distinctes. Alors $u$ est cyclique.

	Soit $x_j$ un vecteur propre associé à la valeur propre $\lambda_j$.

	La matrice de passage entre $(x_1, \dots, x_n)$ et $(x, u(x), \dots, u^{n-1}(x))$ est une matrice de Vandermonde.

	$u(x) = \lambda_1^k x_1 + \dots + \lambda_n^{k}x_n$
	$u^k(x) = \lambda_1^k x + \dots + \lambda_n^{k}x$

	\begin{equation*}
		M = \begin{pmatrix}
			1      & \lambda_1 & \lambda_1^2 & \cdots & \lambda_1^{n-1} \\
			1      & \lambda_2 & \lambda_2^2 & \cdots & \lambda_2^{n-1} \\
			\vdots & \vdots    & \vdots      &        & \vdots          \\
			1      & \lambda_n & \lambda_n^2 & \cdots & \lambda_n^{n-1}
		\end{pmatrix}
	\end{equation*}

	On a que $\det M = 0 $ \ssi les $\lambda_j$ sont disjoints et
	$$ \det M = \prod_{i < j} (\lambda_j-\lambda i)$$
\end{example}

\begin{prop}
	Soit $u \in \LE$ cyclique avec $\dim E = n$, alors le degré du polynôme minimal $\mu_u$ est égal a $n$.
\end{prop}


\begin{proof}
	Soit $x \in E, \ E_{u,x} = \vect \set {u^k(x),  \in \N} = E$.
	Si $\deg \mu_u < n$, cela implique que la famille $\set {x, u(x), \dots, u^{n-1}(x)}$ est liée, ce qui
	contredit $E_{u,x} = E$. En effet si $\deg  \mu_u < n$ alors $\exists (\lambda_k) \ \sum_{k=0}^{\deg \mu_u} \lambda_ u^k = 0$
	En prenant l'image de $x·, \sum_{k=0}^{\deg \mu_u} \lambda_ u^k(x) =0$. Donc $\deg \mu_k \geq n$.

	La famille $\set {x, u(x), \dots , u^{n-1}(x)}$ est une base de $E$. Donc il existe $a_0, \dots, a_{n-1} \in K$ tels que
	$$ u^n(x) = \sum_{k=0}^{n-1} a_k u^k(x) $$
	Montrons que $P(X) = X^n - \sum_{k=0}^{n-1}a_n X^k$ est un polynôme annulateur de $u$. Il suffit de montrer que $P(u)(u^(x)) = 0_E$
	pour tout $0 \leq k \leq n-1$ car $\set {x, \dots, u^{n-1}(x)}$ est une base de $E$. C'est vrai pour $k=0$ d'après (*) %TODO add reference.

	\begin{eqnarray*}
		P(u)(u^k(x)) &=& (P(u) \circ u^k) (x)\\
		&=& (u^ \circ P(u)) (x)
	\end{eqnarray*}
	car les polynômes d'endomorphismes commutent.

	$$ P(u) (u^k(x)) = U^k(\underbrace{P(u)}_{=0} (x)) = 0_E$$

	$P$ est u polynôme annulateur de $u$.
	$\mu_u \mid P$ donc $\deg \mu_u \leq \deg P = n$.
\end{proof}


\begin{prop}
	Soit $u \in \LE$ et $\dim E = n = \deg \mu_u$, alors $u$ est cyclique.
\end{prop}

\begin{proof}
	On sait que $\exists x \in E$, $\mu_u = \mu_{u,x}$.
	$\dim E_{u,x} = \deg \mu_{u,x} = n$
	donc $u$ est cyclique.
\end{proof}




\end{document}
