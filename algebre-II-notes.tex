\documentclass{article}
\usepackage[utf8]{inputenc}


\usepackage{amsmath}
\usepackage{amssymb} 
\usepackage{amsthm}  
\usepackage{dsfont}
\usepackage{mathrsfs}
\usepackage{mathtools}

\usepackage{geometry}

\usepackage{hyperref}        

\usepackage[french]{babel}

\usepackage[shortlabels]{enumitem}

\newcommand{\indep}{\perp\!\!\! \perp}

\theoremstyle{definition} 
\newtheorem{definition}{Définition}

\theoremstyle{definition} 
\newtheorem{prop}{Proposition}

\theoremstyle{definition}
\newtheorem{coro}{Corollaire}

\theoremstyle{plain}
\newtheorem{example}{Exemple}

\theoremstyle{theorem}
\newtheorem{theorem}{Théorème}


\begin{document}

\title{Notes Algèbre II}
\author{Yago iglesias}
\maketitle
\tableofcontents

\section{Anneaux de Polynômes}

\subsection{Construction formelle}




\begin{definition}[Anneau de polynômes]

	Soit $A$ un anneau commutatif. On note $A^{\mathbb{N}}$ l'ensemble des suites presques nulles d'éléments de $A$.
	On muni $A^{\mathbb{N}}$ d'une structure d'anneau en posant:
	\begin{itemize}
		\item $(a_n)_{n \in \mathbb{N}} + (b_n)_{n \in \mathbb{N}} = (a_n + b_n)_{n \in \mathbb{N}}$
		\item $(a_n)_{n \in \mathbb{N}} \cdot (b_n)_{n \in \mathbb{N}} = (c_n)_{n \in \mathbb{N}}$ où $c_n = \sum_{k=0}^{n} a_k b_{n-k}$
	\end{itemize}

	On note $A[X]$ l'anneau commutatif (proposition a montrer si besoin)  $A^{\mathbb{N}}$.

\end{definition}

\begin{prop}
	Soit $A$ un anneau commutatif.
	Soient $P, Q \in A[X]$. Alors $\deg(PQ) \leq \deg(P) + \deg(Q)$.
	De plus, si $A$ est intègre, $\deg(PQ) = \deg(P) + \deg(Q)$.
\end{prop}

\begin{coro}
    $A[X]$ intègre $\iff$ $A$ intègre.
\end{coro}

\subsection{Division euclidienne}

\begin{prop}[Division euclidienne]
	Soit $A$ un anneau commutatif. Soient $P, Q \in A[X]$ avec $Q \neq 0$ et
	$P$ a un coefficient dominant inversible. Alors il existe un unique couple
	$(U, R) \in A[X] \times A[X]$ tel que:
	\begin{itemize}
		\item $P = UQ + R$
		\item $\deg(R) < \deg(Q)$
	\end{itemize}
\end{prop}

\begin{prop}
	Soit $K$ un corps. $K[X]$ est un anneau euclidien et donc principal.
\end{prop}

\begin{proof}
	On commance par noter que $K$ est intègre et donc $K[X]$ aussi.
	Soit $I$ un idéal non nul de $K[X]$. On note $\mathcal{P}$ l'ensemble des degrés des polynômes de $I$:
	\[\mathcal{P} = \{ \deg(P) \mid P \in I \}\]

	$\mathcal{P}$ est non vide et minoré par $0$. Donc il existe $d \in \mathcal{P}$ tel que $d$ est minimal.
	On note $Q$ un polynôme de $I$ de degré $d$. Soit $P \in I$. On réalise la division euclidienne de $P$ par $Q$:
	\[ \exists (U, R) \in K[X] \times K[X] \mid P = UQ + R \text{ et } \deg(R) < \deg(Q) \]
	On a que $R = P - UQ \in I$ et donc $\deg(R) \in \mathcal{P}$. Donc $\deg(R) \geq d$ par définition de $d$.
	Donc $\deg(R) = d$ et donc $R = 0$ car $d$ est minimal. Donc $P = UQ$ et donc $I = <Q>$.
\end{proof}


\begin{example}[Example pathologique]
	$\mathbb{Z}[X]$ n'est pas principal. En effet, soit $I = <2, X>$.
\end{example}

\begin{theorem}[Propriété universelle de l'anneau de polynômes] \label{thm:prop_univ_anneau_poly}
	\label{thm:prop_univ_anneau_poly}
	Soit $A$ un anneau cmmutatif et $B$ un anneau. On considère $f: A \to B$ un morphisme d'anneaux.
	Soit $b \in B$ tel que $f(a) = b$ pour tout $a \in A$. Alors il existe un unique morphisme d'anneaux
	$\tilde{f}: A[X] \to B$ tel que $\tilde{f}(X) = b$ et $\tilde{f}_{|A} = f$.
\end{theorem}

\begin{definition}[Evaluation]
	Soit $A$ un anneau commutatif et $a \in A$. On note $\phi_x A[X] \to A$ le morphisme d'anneaux
	induit par le automorphisme trivial de $A$ en utilisant la propriété universelle de l'anneau de polynômes \ref{thm:prop_univ_anneau_poly}.

	Ce porphisme se correspond à l'évaluation en $a$ : $\phi_x(P) = P(x)$.

\end{definition}

\begin{definition}[Polynôme a plusieurs variables]
	Soit $A$ un anneau commutatif. On définit par récurrence sur $n \in \mathbb{N}$ l'anneau $A[X_1, \dots, X_n]$:
	\begin{itemize}
		\item $A[X_1] = A[X]$
		\item $A[X_1, \dots, X_n] = A[X_1, \dots, X_{n-1}][X_n]$
	\end{itemize}
\end{definition}


\begin{definition}[Anneau factoriel]
	Soit $A$ un anneau commutatif. On dit que $A$ est factoriel si:
	\begin{itemize}
		\item $A$ est intègre
		\item Tout élément non nul de $A$ est inversible ou est produit d'un nombre fini d'éléments irréductibles ( $a = u p_1 \dots p_n$ avec $u$ inversible et $p_i$ irréductible)
		\item La décomposition est unique à l'ordre près et à l'association près.
	\end{itemize}
\end{definition}

\begin{prop}
	Soit $A$ un anneau factoriel. Alors $A[X]$ est factoriel.
\end{prop}


\begin{prop}
	Soit $A$ u n anneau intègre tel que tout élément de $A\setminus\{A^{\times}\}$ est produit d'un nombre fini d'éléments irréductibles, alors les assertions suivantes sont équivalentes:
	\begin{enumerate}
		\item $A$ est factoriel
		\item Si $p \in A$ est irréductible, alors l'idéal $<p>$ est premier
		\item Soient $a,b,c \in A\setminus\{0\}$ tels que $a \mid bc$ et $a$ et $b$ sont premiers entre eux. Alors $a \mid c$ (le lemme de Gauss).
	\end{enumerate}
\end{prop}

\begin{proof}

	3) $\implies$ 2):\\
	Soit $p \in A$ irréductible. On a que $<p> \neq A$ car $p$ est irréductible, donc pas inversible.
	Si $p$ divise $ab$ et ne divise pas $a$, alors $p$ et $a$ sont premiers entre eux car $p$ est irréductible.
	Donc tout diviseur commun de $p$ et de $c$ est soit inversible soit associé à $p$. Donc $p$ divise $c$.
	Donc d'appres 3), $<p>$ est premier.

	\[ ab \in <p>\quad  et \quad a \notin <b> \implies b \in <p> \implies <p> \text{ premier} \]

	Le reste de la preuve est laissé en exercice (je n'arrive pas a bien lire ce qui est écrit dans le cours).


\end{proof}

\end{document}
